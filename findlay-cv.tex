%==============================================================================%
%                                                                              %
%  file:   findlay-cv.tex                                                      %
%  author: Jamie Findlay <jy.findlay@gmail.com>                                %
%                                                                              %
%  This document heavily inspired by the beautiful CVs of Brian Buccola        %
%  (http://brianbuccola.com/) and Tim Rocktäschel (https://rockt.github.io/)   %
%                                                                              %
%==============================================================================%

%============%
%            %
%  PREAMBLE  %
%            %
%============%

\documentclass[11pt,a4paper]{article}

%===============================%
%  Packages and basic settings  %
%===============================%

\newcommand{\name}{Dr Jamie Y. Findlay}

\usepackage{graphicx}

% Font settings
\usepackage[T1]{fontenc}
\usepackage[utf8]{inputenc}
\usepackage{microtype}
\usepackage[normalem]{ulem}
%
\usepackage[sfdefault,scaled=.85]{FiraSans}
\usepackage{newtxsf}
%
\usepackage{setspace}

% Layout
\usepackage{multicol}
\usepackage{enumitem}
\usepackage{layouts}

% Margins and columns
\usepackage[margin=1.2in,lmargin=2.3cm,rmargin=2.3cm,centering,marginparsep=0.8cm]{geometry}

% No indent for new paragraphs
\setlength{\parindent}{0pt}

% Ratios for left and right columns in cvsections and elsewhere
\newcommand{\dateratio}{0.152}
\newcommand{\bodyratio}{0.82}

% Avoid hyphenation across lines
\tolerance=1
\emergencystretch=\maxdimen
\hyphenpenalty=10000
\hbadness=10000

% Some lengths
\newlength{\rulelength}% The length of the rule in section headings
\setlength{\rulelength}{\dateratio\textwidth}
% \addtolength{\rulelength}{11pt}
%
\newlength{\indentlength}% The length of the indent in the \Ref command
\setlength{\indentlength}{29pt}
% \settowidth{\indentlength}{2017}%
% \addtolength{\indentlength}{1em}%

% For manipulating and combining lengths easily
\usepackage{calc}

% Set some colours
\usepackage[svgnames]{xcolor}
\definecolor{gold}{HTML}{c69c6e}
\definecolor{green}{HTML}{008000}
% \definecolor{green}{HTML}{39B54A}
% \definecolor{oxfordblue}{RGB}{4,30,66}
\definecolor{oxfordblue}{HTML}{002147}
\definecolor{oslored}{HTML}{e2231a}
\definecolor{hlinkcolor}{rgb}{.0,.2,.4}

% Set the colour for the header and header rules
\colorlet{headercolor}{oxfordblue}



% Different heading command for research experience sections
\newcommand{\REx}[2]{%
\vspace*{0.1\baselineskip}%
% \rule[0.4ex]{\rulelength}{1pt}\hspace*{11pt}%
{\large\textbf{#1}\hfill\textnormal{[#2]}}%
\vspace*{0.5\baselineskip}%
}
%Slightly nicer look but fits less in
% \newcommand{\REx}[2]{%
% \vspace*{0.5\baselineskip}%
% % % \rule[0.4ex]{\rulelength}{1pt}\hspace*{11pt}%
% {\large\textbf{#1}\hfill\textnormal{[#2]}}%
% \vspace*{0.7\baselineskip}%
% }

% Headers, footers, and page styles
\usepackage{titleps}
\newpagestyle{main}[\footnotesize]{%
  \headrule
  \sethead[][][\name]%
          {}{}{\name}
  \setfoot[][\thepage][\scriptsize \today]
          {}{\thepage}{\scriptsize \today}
}
\newpagestyle{first}[\footnotesize]{%
  \sethead{}{}{}
  \setfoot{}{}{\scriptsize \today}
}
\pagestyle{main}

% Packages for CV environment
\usepackage{array}% required for defining newcolumntype with custom vrule
\usepackage{longtable}% normal \tabular environment does not allow page breaks
\setlength{\LTpre}{0pt}% glue before longtable
\addtolength{\LTpost}{0pt}% glue after longtable

% Main body sections
\newcolumntype{L}{>{\raggedleft\footnotesize}p{\dateratio\textwidth}}
\newcolumntype{R}{p{\bodyratio\textwidth}}
%
\newenvironment{cvsection}{%
  \setlength{\extrarowheight}{0.70ex}
  \begin{longtable}[l]{@{} L R @{}}
  %% \begin{longtable}[l]{@{}L@{} @{}R@{}} % Brian's fix of 14/03/21
}{%
  \end{longtable}
}

% Lists for reviews etc.
\newlength{\squish}
\setlength{\squish}{-14.275pt}
%% 14pt is the sum of the default values of \topsep (8pt), \partopsep (2pt), and
%% \parsep (4pt). This isn't quite right: the list is too low. But if you use
%% the values these lengths have in this document (9pt, 3pt, and 4.5pt), the
%% list is too high up ... Hence the super-hacky technique.

\newenvironment{reviewlist}
{%
\vspace*{\squish}%
\begin{itemize}[noitemsep,label={},nosep,left=0pt .. \parindent]%
}
{%
\end{itemize}
}

% \newenvironment{languagelist}
% {\begin{reviewlist}}
% {\end{reviewlist}}

% Allow starred command names
\usepackage{suffix}

% Hyperlinks
\usepackage[breaklinks,hidelinks]{hyperref}
\hypersetup{%
colorlinks=true,%
urlcolor=hlinkcolor,%
linkcolor=hlinkcolor,%
% citecolor = black%
}

\urlstyle{same}


% Enable margin notes everywhere
\usepackage{marginnote}

% Force marginnotes to be on the LHS
\usepackage{etoolbox}
\makeatletter
\patchcmd{\@mn@margintest}{\@tempswafalse}{\@tempswatrue}{}{}
\patchcmd{\@mn@margintest}{\@tempswafalse}{\@tempswatrue}{}{}
\reversemarginpar
\makeatother

% Allow tables to fit to line width
\usepackage{tabularx}

% Icons
\usepackage{fontawesome} % Provides icon commands
\newcommand{\icon}[1]{\raisebox{-.2\dp\strutbox}{#1}} % Aligns icons vertically
\newcommand{\oicon}[1]{\raisebox{-0.5\dp\strutbox}{#1}} % For the ORCID icon


%==========%
%  Macros  %
%==========%

% For references list
\newcounter{RefNo} % Counter for publications

% Just adds a number
\newcommand{\refmark}{%
  \refstepcounter{RefNo}%
  \marginnote{[\theRefNo]}
}

% The \Reff command takes three arguments: a x-ref label, a date, and the bibliographical info
\newcommand\Reff[3]{%
  \refmark%
  \hangindent=\indentlength%
  \label{#1}%
  \textbf{#2}%
  \hspace*{1em}%
  #3%
  \vspace*{0.7ex}%
  }

% A starred version for presentations, which have a longer date
\WithSuffix\newcommand\Reff*[3]{%
  \hangindent=\indentlength%
  \refmark%
  \label{#1}%
  \textbf{#2}%
  \hspace*{0.5em}%
  #3%
  \vspace*{0.7ex}%
}

%% Commands for hyperlinks

% Files stored in my webspace -- set the first part to wherever things live at the moment
\newcommand{\mylink}[2]{\href{http://users.ox.ac.uk/~sjoh2787/#1}{#2}}
% Starred version with underlining
\WithSuffix\newcommand\mylink*[2]{\href{http://users.ox.ac.uk/~sjoh2787/#1}{\uline{#2}}}

% For links to other URLs
\newcommand{\linkout}[2]{\href{#1}{#2}}
% Version with typewriter font and square brackets
\newcommand{\publinkout}[2]{\href{#1}{\texttt{[#2]}}}

% Link commands for presentations
\newcommand{\handout}[1]{\mylink{#1}{\texttt{[handout]}}}
\newcommand{\poster}[1]{\mylink{#1}{\texttt{[poster]}}}
\newcommand{\slides}[1]{\mylink{#1}{\texttt{[slides]}}}

% Command for open access links, which takes a colour argument -- NOT CURRENTLY IN USE
\newcommand{\oa}[2]{\marginnote{\hspace*{-1em}{\href{#2}{\texttt{[\textcolor{#1}{paper}]}}}}}

% Puts square brackets around label numbers when x-ref'd
\newcommand{\sref}[1]{[\ref{#1}]}

% A note macro
\newcommand{\note}{\emph{Note: }}

% Long dates
\newcommand{\longdate}[1]{\parbox[t]{\dateratio\textwidth}{\raggedleft
#1}}

% Changing font for \LaTeX command because it looks gross in Fira Sans
\WithSuffix\newcommand\LaTeX*{%
  {%
  \fontfamily{cmr}%
  \selectfont\LaTeX%
  }%
}

% Doing the same for \TeX for consistency
\WithSuffix\newcommand\TeX*{%
  {%
  \fontfamily{cmr}%
  \selectfont\TeX%
  }%
}

% Additional info notes for main body entries
\newcommand{\Note}[2]{%
\parbox[t]{\bodyratio\textwidth}{#1\\[-0.25em]{\footnotesize #2}}%
}

% Labels for the left-hand column in main body entries
\newcommand{\Label}[1]{%
\textnormal{#1}%
}

% Subheadings in research experience
\newcommand{\researchsubhead}[1]{%
\textsc{#1}:%
}

% SECTION COMMANDS
\newcommand{\cvheading}[1]{\noindent{{\color{headercolor}\rule[0.4ex]{\rulelength}{2pt}\hspace*{9pt} \Large #1}}\vspace*{0.5\baselineskip}}

\newcommand{\cvsubhead}[1]{\noindent\hspace*{\rulelength}\hspace*{9pt} \textsc{#1}\vspace*{0.25\baselineskip}}

\newcommand{\rulesubhead}[1]{\noindent{\color{headercolor}\rule[0.4ex]{\rulelength}{1pt}\hspace*{9pt} {#1}}\vspace*{0.25\baselineskip}}

\newcommand{\cvsubsubhead}[1]{\noindent\hspace*{\rulelength}\hspace*{9pt} \textit{#1}\vspace*{0.25\baselineskip}}

\newcommand{\pubsubhead}[1]{\noindent\hspace*{\rulelength}\hspace*{9pt} \textsc{#1}\vspace*{0.25\baselineskip}}



% Contact info environment
\newcommand{\ContactInfo}[1]{%
\parbox[c]{\hsize}{\raggedleft\footnotesize\it%
#1}%
}

% Course convenor annotation
\newcommand{\conv}{
% \reversemarginpar
% \marginnote{%
$\mathcal{C}''\hfill%
% }
% \normalmarginpar
}

%========%
%        %
%  BODY  %
%        %
%========%

\frenchspacing

\begin{document}

\thispagestyle{first}

\begin{tabularx}{\linewidth}{@{}lX@{}}
{\Huge\textbf{\name}}%
&\ContactInfo{%
%% Centre for Linguistics and Philology\\
%% Walton St., Oxford, OX1 2HG\\
% United Kingdom\\
\href{mailto:jamie.findlay@iln.uio.no}{{\icon{\faEnvelopeO}\ jamie.findlay@iln.uio.no}}\\
  \href{http://jyfindlay.com}{\icon{\faChain}\ jyfindlay.com}\\
  \href{http://www.linkedin.com/in/findlayjy}{\icon{\faLinkedinSquare}\ findlayjy}\\
  \href{https://github.com/findlayjy}{\icon{\faGithub}\ findlayjy}\\
  \href{https://scholar.google.com/citations?user=Q2v46FwAAAAJ}{\icon{\faGraduationCap}\ Google Scholar profile}\\ \href{https://orcid.org/0000-0002-7700-1525}{\oicon{\includegraphics[height=2ex]{orcid-hirez}} orcid.org/0000-0002-7700-1525}
}
\end{tabularx}

%%==========================================================================%%

\cvheading{Research interests}

\begin{cvsection}
    &
     Syntax, semantics, grammatical theory and the syntax-semantics interface (especially Lexical Functional Grammar and Glue Semantics), pragmatics, (critical) discourse analysis, language \& gender, language \& sexuality.
\end{cvsection}

%%==========================================================================%%

\cvheading{Academic employment}

\begin{cvsection}
  \longdate{01/04/2021--\\present} & \textbf{Post-Doctoral Research Fellow in Semantics}, University of Oslo.\\
  \longdate{01/12/2019--\\31/03/2021} & \textbf{Outreach \& Access Officer} (for linguistics), University of Oxford.\\
  \longdate{01/10/2019--\\31/03/2021} & \textbf{Departmental Lecturer in Syntax}, University of Oxford.\\
  \longdate{01/10/2019--\\ 30/09/2020}& \textbf{Stipendiary Lecturer in Linguistics}, Jesus College, University of Oxford.\\
  \longdate{01/10/2018--\\ 30/09/2019} & \textbf{Stipendiary Lecturer in Linguistics}, St Hugh's College, University of Oxford.
\end{cvsection}

%%==========================================================================%%

\cvheading{Education}

\begin{cvsection}
  \longdate{11/10/2015--\\06/05/2019} &
  \Note{\textbf{D.Phil. in Comparative Philology \& General Linguistics.}\\
    University of Oxford.}%
  {\textsc{Thesis title}: \emph{Multiword expressions and the lexicon} \mylink{findlay-thesis.pdf}{\texttt{[pdf]}}.  \\[-0.15em]
    \textsc{Supervisors}: Ash Asudeh, Mary Dalrymple. %[-0.5em]
    \textsc{Examiners}: Stephen Pulman, Tom Wasow. }
  \\
  \longdate{09/10/2012--\\30/06/2014}&
  \Note{\textbf{M.Phil. (with distinction) in General Linguistics \& Comparative Philology.}\\
    University of Oxford.}%
  {\textsc{Thesis title}: \emph{The prepositional passive: a Lexical Functional account}.\\[-0.15em]
    \textsc{Supervisor}: Mary Dalrymple. }
  \\
  \longdate{01/09/2011--\\06/12/2011} & \Note{Graduate Programme in Linguistics.\\
    McGill University.} {Attended for one term. GPA 3.92.
  }
  \\
  \longdate{09/10/2007--\\30/06/2011} & \parbox[t]{\bodyratio\textwidth}{\textbf{B.A. (with First Class Honours) in Modern Languages (French) \& Linguistics.}\\
    University of Oxford.}
\end{cvsection}


%%==========================================================================%%

\cvheading{Teaching qualifications}

\begin{cvsection}
  05/12/2017 & \Note{%
    \textbf{SEDA-PDF Supporting Learning award.}%
  }%
  {Mapped to Descriptor 1 of the UK Professional Standards Framework: equivalent to Associate Fellow\\[-0.5em] of the Higher Education Academy.\\
    My portfolio submission is available online
    \mylink*{portfolio-JYF-submission.pdf}{here}. }%
\end{cvsection}

%%==========================================================================%%

\cvheading{Other professional experience}

\begin{cvsection}
  \longdate{01/07/2012--01/10/2018} & \Note{\textbf{Linguistic
      Consultant}, Freelance.}{Expert consultancy services in and around general linguistics and NLP. This involved data annotation, resource\\[-0.5em]
    mining and compilation, data analysis, and creative copywriting. I have worked with researchers at the University\\[-0.5em]
    of Oxford and at Queen Mary University London, as well as with private enterprises such as TheySay Analytics.}\\
  \longdate{01/12/2014--30/09/2015} & \textbf{Part-time Research Assistant},
  Oxford English Dictionary, Oxford University Press.
\end{cvsection}

%%==========================================================================%%
\newpage

\cvheading{Research experience}

\note\ My publications and talks are given on pp.~\pageref{scholarship}ff. This section gives details of some recent projects, with (clickable) cross-references to related publications and talks.\\

\REx{Multiword expressions and the lexicon}{syntax, semantics}
\begin{cvsection}
  \researchsubhead{Overview} &%
  My D.Phil. thesis (\sref{thesis}) gives an analysis of multiword expressions
  (MWEs), especially idioms. It argues against approaches which conceive of
  idioms like \emph{spill the beans} as being made up special versions of
  \emph{spill} and \emph{beans}, and in favour of approaches which reaffirm the
  mixed nature of MWEs as partially word-like and partially phrase-like. It then
  develops a concrete example of the latter kind of theory, integrating insights
  from Tree-Adjoining Grammar into the Lexical Functional Grammar architecture.%
  \\
  \researchsubhead{Outcomes} &%
  I have presented this work at peer-reviewed conferences for computational
  linguists (\sref{eacl2017-talk}), theoretical syntacticians
  (\sref{lfg2017-talk}), and for more general linguistics audiences
  (\sref{dgfs2018-talk}). The work has also appeared in the proceedings of two
  of these conferences (\sref{lfg2017}, \sref{eacl2017}). I have also organised
  a collaboration with researchers at the Goethe-Universit\"{a}t Frankfurt am
  Main, where I conducted a research visit in June 2018.
  \\
  \researchsubhead{Future work} &%
  Sascha Bargmann, Manfred Sailer and I are developing a semantic theory of the
  `extended' use of idioms, exemplified in e.g. \textit{Once this cat is out of
    the bag, a lot of people are going to get scratched}, where the underlying
  metaphor is used to make sense of a figurative continuation
  (\sref{creative-power-metaphor-poster}, \sref{europhras2019-talk},
  \sref{manchester-seminar}, \sref{fmm2018}).
\end{cvsection}

\REx{Mapping Theory}{syntax, semantics}
\begin{cvsection}
  \researchsubhead{Overview} &%
  Mapping Theory is a theory within LFG which handles the linking of grammatical
  functions in the syntax with argument positions in the semantics. My work on
  this topic addresses a previous lack of explicit formalisation in this area,
  by formalising the version of Mapping Theory developed by Anna Kibort. This
  work has proved influential, and so-called `Kibort-Findlay Mapping Theory' now
  constitutes the state of the art.
  \\
  \researchsubhead{Outcomes} &%
  My original approach was published in the \textit{Journal of Language
    Modelling} (\sref{arg-str-paper}), while a formally cleaner version was
  presented at the LFG Conference in 2020 (\sref{lfg2020-talk}) and will appear
  in the proceedings (\sref{lfg2020}). Anna Kibort and I are also writing a
  chapter on Mapping Theory for the upcoming \textit{Handbook of Lexical
    Functional Grammar} (\sref{arg-str-handbook-chapter}).
  \\
  \researchsubhead{Future work} &%
  Going forward, I intend to improve the cross-linguistic coverage of the
  theory, testing how well the formal system I have developed can cope with the
  different argument structures and valency altering operations of the world's
  languages.
\end{cvsection}

\REx{Discourses of homosexuality}{sociolinguistics, corpus linguistics}
\begin{cvsection}
  \researchsubhead{Overview} &%
  I conducted a short research project using corpus linguistic and critical discourse analysis tools to examine the UK House of Lords debates on same-sex marriage in 2013. This contributed to a growing literature both on debates around same-sex marriage legislation and also on implicit homophobia.%
\\
  \researchsubhead{Outcomes} &%
  My findings were reported at the IGALA conference in 2014 (\sref{igala2014-talk}) and published in the \textit{Journal of Language and Sexuality} (\sref{house-of-lords-paper}).%
\\
  \researchsubhead{Future work} &%
  I would like to continue this research by examining parliamentary debates from other countries where discursive norms differ in interesting ways, such as the focus on family values and religion in Uganda or the USA.
\end{cvsection}

% %%==========================================================================%%
\newpage

\cvheading{Teaching}

\begin{cvsection}
  &
  I have taught \textbf{syntax}, \textbf{semantics \& pragmatics}, \textbf{sociolinguistics}, \textbf{psycholinguistics}, and \textbf{general linguistics} at both the introductory and advanced undergraduate level. At the graduate level, I have led courses in \textbf{syntax}, and TA-ed for courses in \textbf{syntax}, \textbf{phonology}, and \textbf{formal foundations of linguistics} (the latter being an introduction to set theory, logic, and formal language theory for linguists). This teaching experience covers small group and tutorial settings, larger classes, and lectures, including designing tutorial courses and lecture series. I hold a SEDA-PDF Supporting Learning award (see above).\\
  & In this section, years refer to the academic calendar (e.g. 2017 = the
  2017/18 academic year).
\end{cvsection}

\rulesubhead{University of Oslo}

\cvsubhead{First year undergraduate level}

\begin{cvsection}
  2021 & \Note{\textbf{Sociolinguistics} (seminars -- group size: ??)}{Half of
          LING1113, `Psycholinguistics \& Sociolinguistics'}
\end{cvsection}

\rulesubhead{University of Oxford}

\cvsubhead{Graduate level}

\cvsubsubhead{As primary instructor}
\begin{cvsection}
    2019--2020        & \textbf{Contemporary Syntactic Theory} (lectures and classes -- group size: \textasciitilde{}12).\\
    2019--2020        & \textbf{Introduction to Syntax} (lectures -- group size: \textasciitilde{}25).
\end{cvsection}

\cvsubsubhead{As TA}
\begin{cvsection}
  2017--2018 & \Note{ \textbf{Formal Foundations of Linguistics} (classes --
    group size: \textasciitilde{}25).}
  {An introduction to set theory, logic, and formal language theory for linguists.}\\
  2015--2016 & \textbf{Introduction to Syntax} (classes -- group
  size: \textasciitilde{}25).\\
  2014 & \textbf{Introduction to Phonology} (classes -- group size: 12).
\end{cvsection}

\cvsubhead{Advanced undergraduate level}

\begin{cvsection}
    2014--2016, 2019--2020
                & \textbf{Syntax} (tutorials -- group size: 1--3).\\
  2018--2019  & \Note{\textbf{General Linguistics} (tutorials -- group size: 2--3).}
              {A survey module which covers the development of
                linguistic theory since the 19th century.}\\
    2017--2019
                &	\textbf{Sociolinguistics} (tutorials -- group size: 1--3).\\
    2018  & \textbf{Psycholinguistics} (tutorials -- group size: 1--3).\\
    2014--2017  & \textbf{Semantics \& Pragmatics} (tutorials -- group size: 1--3).
  \end{cvsection}

\cvsubhead{First-year undergraduate level}
\begin{cvsection}
    2014, 2018--2020
                & \Note{%
                \textbf{Grammatical Analysis} (classes -- group size: 3--6).}
                {An introduction to both syntax and morphology.}\\
    2014, 2018--2019
                &	\Note{%
                \textbf{General Linguistics} (tutorials -- group size: 2--3).}
                {An introductory course which covers topics in general linguistics proper, along with psycholinguistics,\\[-0.5em] semantics \& pragmatics, sociolinguistics, and historical  linguistics.}\\
    2018        & \textbf{Psycholinguistics} (lectures -- group size: \textasciitilde{}50).\\
    2017        & \textbf{Semantics \& Pragmatics} (lectures -- group size: \textasciitilde{}50).
\end{cvsection}

\cvsubhead{Irregular classes}
\begin{cvsection}
    2016, 2019	& \Note{%
                  \textbf{Language and identity: the view from linguistics} (class -- group size: \textasciitilde{}6).}
                  {An introduction to sociolinguistics for first year English language and literature undergraduates.\\[-0.25em]
                  Handout available \mylink*{socio_presentation.pdf}{here}.}\\
    2017        & \textbf{Corpus methods in sociolinguistics} (graduate class -- group size: 3).\\
    2016--2017, 2019 & \Note{%
                  \textbf{\LaTeX*\ workshops}: `\LaTeX*\ basics' and `Going further with \LaTeX*'.}
                  {Two workshops to introduce graduate students in linguistics to \LaTeX*.\\[-0.25em]
                  Handouts available \mylink*{latex-basics.pdf}{here} and \mylink*{latex-going-further.pdf}{here}.}
\end{cvsection}

% %%==========================================================================%%
% \newpage
\cvheading{Supervision}

% \rulesubhead{University of Oxford}

\cvsubhead{Graduate supervision (University of Oxford)}

\begin{cvsection}
  2021 & Robert Flick. Extended essay in syntax. Title: \textit{Distinguishing
    middles and passives in Hungarian: a numerical information structure
    approach}. M.St. in Linguistics, Philology and Phonetics. \\

  2021 & Kamran Sharifi. Extended essay in syntax. Title \textit{``Crazy
    syntax'': examining the role of the synchronic/diachronic distinction in
    syntactic theory}. M.St. in Linguistics, Philology and Phonetics.\\

  2021 & Tianxin Tu. Extended essay in syntax. Title: \textit{The Mandarin
    Chinese \emph{ba} construction revisited: an LFG account}. M.St. in Linguistics, Philology and Phonetics.\\

  2021 & Xiulin Yang. Extended essay in syntax. Title: \textit{The ambiguous
    binding of \emph{ziji}: an LFG analysis}. M.St. in Linguistics, Philology and Phonetics.\\

  2021 & Tommy Zhang. Extended essay in syntax. Title: \textit{The concept of
    Functional-Argument Biuniqueness in LFG \& Glue and additional evidence
    from reflexives and complex predicates}. M.St. in Linguistics, Philology and Phonetics.\\

  2020 & Chenyu Fang. Extended essay in syntax. Title: \textit{An LFG analysis
    of anaphoric reflexive \emph{(ta)-ziji} in Mandarin Chinese}. M.St. in Linguistics, Philology and Phonetics.
\end{cvsection}

\cvsubhead{Undergraduate supervision (University of Oxford)}

\begin{cvsection}
  2020--2021 & Caitlyn McDermott. Linguistic project (thesis with a large data
  collection\slash analysis component). \textit{What are they talking about? A discourse analysis of US press coverage of mass shootings}. BA in Psychology \& Linguistics.\\
  %
\end{cvsection}

%% ==========================================================================%%

% \cvheading{Mobility}

% \begin{cvsection}
%   June 2018 & \textbf{Visiting researcher}, Goethe-Universit\"{a}t Frankfurt am Main, Department of English and American Studies.
% \end{cvsection}

%%==========================================================================%%

\cvheading{Grants and awards}

\begin{cvsection}
  May 2018   & \Note{\textbf{Scatcherd European Scholarship}.}{Value: £1,000. Awarded, with the next award, for a research visit to the Goethe-Universit\"{a}t Frankfurt am\\[-0.5em] Main.}\\
  %
  Mar. 2018   & \Note{\textbf{Santander Academic Travel Award}.}{Value: £1,000.}\\
  %
  \longdate{%
    Oct. 2015--\\June 2017}
  % 2015--2017
  & \Note{\textbf{Professor Paul Slack Scholarship}, Linacre College, University of Oxford.}
  {Total value: \textasciitilde{}£14,000 (fees for first 2 years of D.Phil.).}\\
  %
  \longdate{%
    Oct. 2015--\\June 2017}
  % 2015--2017
  & \Note{\textbf{AHRC Studentship}, Faculty of Linguistics, Philology \& Phonetics, University of Oxford.}
  {Total value: £28,353 (living costs for first 2 years of D.Phil.).}\\
  %
  July 2014 & \Note{\textbf{George Wolf Prize}, Faculty of Linguistics, Philology \& Phonetics, University of Oxford.}{Awarded for best overall performance in the M.Phil. or M.St. degree.}\\
  %
  June 2014 & \textbf{Graduate student essay prize}, International Gender and Language Association.\\
  %
  \longdate{%
    Oct. 2012--\\June 2014}
  % 2012--2014
  & \Note{\textbf{AHRC Studentship}, Faculty of Linguistics, Philology \& Phonetics, University of Oxford.}
  {Total value: \textasciitilde{}£38,000 (fees and living costs for M.Phil.).}\\
  %
  \longdate{Sept. 2011--\\June 2013}
  & \Note{\textbf{McCall McBain Fellowship}, McGill University.}
  {Total value: \textasciitilde{}CAN\$62,000 (fees and living costs for 2-year MA degree); partially declined.}\\
  %
  \longdate{%
  Oct. 2008--\\June 2011}
  % 2008--2011
  & \Note{\textbf{Casberd Scholarship}, St John's College, University of Oxford.}
  {Awarded for achieving a distinction in first year examinations. Total value: \textasciitilde{}£900.}
\end{cvsection}

% %%==========================================================================%%
\newpage
\cvheading{Service}

\cvsubhead{To the profession}
\begin{cvsection}
  \Label{Committee member} & Executive Committee, International Lexical Functional Grammar Association (ILFGA) (Sept.~2021--present).\\
  \Label{Editor} & \textit{Proceedings of the LFG Conference}~(2021--present).\\
\Label{Grant reviewer} &
                    Narodowe Centrum Nauki (Polish National Science Centre)~(2019).\\
\Label{Manuscript reviewer} &
                \begin{reviewlist}
                \item \textit{Corpora and Discourse Studies}~(2018),
                \item \textit{Journal of Language and Sexuality}~(2019),
                \item \textit{Journal of Linguistics}~(2015, 2017, 2019),
                \item \textit{Poznan Studies in Contemporary Linguistics}~(2019),
                \item \textit{Proceedings of the LFG Conference}~(2018--2020),
                \item  \textit{Syntax}~(2020).
                \end{reviewlist}
                 \\[\squish]
\Label{Abstract reviewer}   &
                \begin{reviewlist}
                \item \textit{Conference of the Student Organisation of Linguistics in Europe (ConSOLE)}~(2018),
                \item \textit{International Lexical Functional Grammar Conference} (2019--2021).
                \end{reviewlist}
                \\[\squish]
\Label{Marker}      & UK Linguistics Olympiad~(since 2014).\\
\Label{Copy editor} & \textit{Journal of Language Modelling} (since 2017).
\end{cvsection}
\vspace{0.2\squish}
\cvsubhead{To the department}
\begin{cvsection}
  Oct. 2019--Mar. 2021 & \Note{\textbf{Committee member}, Athena Swan Self Assessment Team, Faculty of Linguistics, Philology \& Phonetics, University of Oxford.}{Athena Swan is a gender equality certification for higher education institutes in the UK.}\\
  Oct. 2018--Mar. 2021 & \textbf{Committee member}, Undergraduate Studies Committee, Faculty of Linguistics, Philology \& Phonetics, University of Oxford.\\
  Oct. 2019--Sept. 2020 & \Note{\textbf{Personal tutor and Linguistics Organising Tutor}, Jesus College.}{Providing pastoral support and coordinating teaching for undergraduate linguistics students.}\\
  \longdate{Jan.--Mar. 2018, \\Jan.--June 2020} & \textbf{Convenor}, Syntax Working Group, University of Oxford.\\
  Dec. 2015--Dec. 2019 & \textbf{Admissions interviewing} for undergraduate degrees involving linguistics, University of Oxford.\\
  Oct. 2018--Sept. 2019  & \textbf{Personal tutor and Linguistics Organising Tutor}, St Hugh's College.\\
  Oct. 2017--June 2018 & \textbf{Co-founder and co-convenor}, D.Phil. workshop, Faculty of Linguistics, Philology \& Phonetics, University of Oxford.
    % 2008--2009  & President, Oxford University Linguistics Society.\\
    % 2007--2009  & Student representative, Faculty of Linguistics, Philology \&
                % Phonetics, University of Oxford.\\
\end{cvsection}


%%==========================================================================%%
% \newpage
\cvheading{Languages}

\begin{cvsection}
  \Label{Natural} & English~(native), French~(advanced), Latin~(intermediate),
  German~(beginner), Norwegian~(beginner).
  \\
  \Label{Programming}  &     Bash, Python, Emacs Lisp.\\
  \Label{Markup} & \LaTeX*\slash Bib\TeX*, HTML\slash CSS.
\end{cvsection}


%%==========================================================================%%

\cvheading{Professional Memberships}

\begin{cvsection}
  Since 2017 & Association for Computational Linguistics (ACL).\\
  Since 2014 & International Lexical Functional Grammar Association (ILFGA).\\
  Since 2012 & Canadian Linguistic Association (CLA\slash ACL).\\
  Since 2011 & Linguistics Association of Great Britain (LAGB).\\
  Since 2011 & International Gender and Language Association (IGALA).\\
\end{cvsection}

%%==========================================================================%%

\newpage
\newcounter{dummy}
\refstepcounter{dummy} % This makes the hyperlinks from above jump to the correct page


\cvheading{Scholarship} \label{scholarship}

Where possible, clicking on a title will take you to an openly accessible version of that item. For conference presentations, I provide a link
to the slides, handout, or poster, as appropriate.\medskip

Number of citations is given per Google Scholar, and does not include
self-citations.

\subsection*{Thesis}

\Reff{thesis}{2019}{Jamie Y. Findlay.
  \linkout{https://ora.ox.ac.uk/objects/uuid:502e8ca5-02f7-4be4-8778-cd89364ba670/download_file?file_format=pdf&safe_filename=findlay-thesis-deposit.pdf&type_of_work=Thesis}{\textit{Multiword
      expressions and the lexicon.}} D.Phil. thesis, University of Oxford.}

\subsection*{Articles in Refereed Journals and Books}
%
\Reff*{arg-str-handbook-chapter}{In prep.}{Jamie Y. Findlay, Anna Kibort and
  Roxanne Taylor. Argument structure and Mapping Theory. In Mary Dalrymple
  (ed.), \emph{Handbook of Lexical Functional Grammar}. Language Science Press.}

\Reff*{lfg-tag-chapter}{In prep.}{Jamie Y. Findlay. LFG and Tree Adjoining
  Grammar. In Mary Dalrymple (ed.), \emph{Handbook of Lexical Functional
    Grammar}. Language Science Press.}

\Reff*{meaning-in-lfg-chapter}{2021}{Jamie Y. Findlay.
  \mylink{festschrift-chapter.pdf}{Meaning in LFG}. In I. Wayan Arka, Ash
  Asudeh, and Tracy Holloway King (eds.), \emph{Modular design of grammar:
    linguistics on the edge}, 340--374. Oxford University Press.}

\Reff{lfg-chapter}{2019}{Mary Dalrymple and Jamie Y. Findlay.
  \mylink{dalrymple_findlay-2019-lexical_functional_grammar.pdf}{Lexical
    Functional Grammar}. In Andr\'{a}s Kert\'{e}sz, Edith Moravcsik, and Csilla
  R\'{a}kosi (eds.), \emph{Current approaches to syntax: a comparative
    handbook}, 123--154. De Gruyter Mouton. DOI:
  \href{https://doi.org/10.1515/9783110540253-005}{10.1515/9783110540253-005}.
  \hfill[1 citation]}

\Reff{house-of-lords-paper}{2017}{Jamie Y. Findlay.
  \mylink{house_of_lords.pdf}{Unnatural acts lead to unconsummated marriages:
    discourses of homosexuality within the {House of Lords} debate on same-sex
    marriage}. \emph{Journal of Language and Sexuality} 6(1), 30--60. DOI:
  \href{https://doi.org/10.1075/jls.6.1.02fin}{10.1075/jls.6.1.02fin}. \hfill[16
  citations]}

\Reff{arg-str-paper}{2016}{Jamie Y. Findlay.
  \linkout{http://jlm.ipipan.waw.pl/index.php/JLM/article/view/171/14}{Mapping
    theory without argument structure}. \emph{Journal of Language Modelling}
  4(2), 293--338. DOI:
  \href{http://dx.doi.org/10.15398/jlm.v4i2.171}{10.15398/jlm.v4i2.171}.
  \hfill[34 citations]}

\Reff{bmj-paper}{2013}{Janak Bechar, Jamie Y. Findlay, and Joseph Hardwicke.
  \mylink{medical_terminology.pdf}{Understanding medical words of Greek and
    Latin origin}. \emph{Student BMJ} 21. DOI:
  \href{https://doi.org/10.1136/sbmj.f7394}{10.1136/sbmj.f7394}.}

\subsection*{Articles in Refereed Conference Proceedings}

\Reff{lfg2020}{2020}{Jamie Y. Findlay.
  \href{http://web.stanford.edu/group/cslipublications/cslipublications/LFG/LFG-2020/lfg2020-findlay.pdf}{Mapping
    Theory and the anatomy of a verbal lexical entry}. In Miriam Butt and Ida
  Toivonen (eds.), \emph{Proceedings of the LFG'20 Conference}, 127--147.
  Stanford, CA: CSLI Publications.}

\Reff{lfg2017}{2017}{Jamie Y. Findlay.
  \href{http://web.stanford.edu/group/cslipublications/cslipublications/LFG/LFG-2017/lfg2017-findlay.pdf}{Multiword
    expressions and lexicalism}. In Miriam Butt and Tracy Holloway King (eds.),
  \emph{Proceedings of the LFG'17 Conference}, 209--229. Stanford, CA: CSLI
  Publications. \hfill[1 citation]}

\Reff{eacl2017}{2017}{Jamie Y. Findlay.
  \linkout{https://www.aclweb.org/anthology/W17-1709.pdf}{Multiword expressions
    and lexicalism: the view from LFG}. In \emph{Proceedings of the 13th
    Workshop on Multiword Expressions (MWE 2017)}, 73--79. Association for
  Computational Linguistics.
  DOI:~\href{https://doi.org/10.18653/v1/W17-1709}{10.18653/v1/W17-1709}.
  \hfill[2 citations]}

\Reff{headlex}{2016}{Jamie Y. Findlay.
  \linkout{http://web.stanford.edu/group/cslipublications/cslipublications/HPSG/2016/headlex2016-findlay.pdf}{The
    prepositional passive in Lexical Functional Grammar}. In Doug Arnold, Miriam
  Butt, Berthold Crysmann, Tracy Holloway King, and Stefan M\"uller (eds.),
  \emph{Proceedings of the Joint 2016 Conference on Head-driven Phrase Structure
    Grammar and Lexical Functional Grammar}, 255--275. Stanford, CA: CSLI
  Publications. \hfill[6 citations]}
\newpage

\subsection*{Refereed Presentations}

\Reff*{iccg11-talk}{20 Aug., 2021}{Constructions in LFG. Presented online at the
  Workshop on Constructional Approaches in Formal Grammar at the 11th
  International Conference on Construction Grammar (ICCG11), University of
  Antwerp. \slides{iccg-handout-findlay.pdf}}

\Reff*{lfg2020-talk}{25 June, 2020}{Lexical Mapping Theory and the anatomy of a
  lexical entry. Presented online at the 25th International Lexical Functional
  Grammar Conference, University of Oslo. \slides{jfindlay-lfg2020-slides.pdf}}

\Reff*{creative-power-metaphor-poster}{30 Mar., 2019}{Extended idioms as
  metaphors. Joint work with Sascha Bargmann and Manfred Sailer. Poster
  presented at the Conference on the Creative Power of Metaphor, Worcester
  College, University of Oxford. \poster{FBS-poster.pdf}}

\Reff*{europhras2019-talk}{24 Jan., 2019}{Why the
  butterflies in your stomach can have big wings: combining formal and cognitive
  theories to explain productive extensions of idioms. Joint work with Sascha
  Bargmann and Manfred Sailer. Presented at the International Conference of the
  Europ\"{a}ische Gesellschaft f\"{u}r Phraseologie (EUROPHRAS 2019): Productive
  Patterns in Phraseology, Santiago de Compostela, Spain. \slides{europhras19.pdf}}

\Reff*{lfg2018-talk}{18 July,
  2018}{Lexical and grammatical
  meaning: attributive adjectives in French. Joint work with Hannah Senior.
  Presented at the 23rd International Lexical Functional Grammar Conference,
  University of Vienna. \slides{findlay-senior-lfg2018-slides.pdf}}

\Reff*{dgfs2018-talk}{8 Mar., 2018}{Conspiracy theories: the problem with
  lexical approaches to idioms. Presented at the Workshop on One-to-Many
  Relations in Morphology, Syntax and Semantics at the 40th Annual Meeting of
  the Deutsche Gesellschaft f\"{u}r Sprachwissenschaft (DGfS 2018),
  Universit\"{a}t Stuttgart. \slides{findlay-dgfs-slides.pdf}}

\Reff*{lagb2017-talk}{6 Sept., 2017}{The logic of
  Korean honorification. Joint work with Yoolim Kim. Presented at the 2017
  Annual Meeting of the Linguistics Association of Great Britain (LAGB 2017),
  University of Kent, Canterbury. \slides{kim-findlay-lagb.pdf}}

\Reff*{lfg2017-talk}{27 July, 2017}{Multiword expressions and lexicalism.
  Presented at the 22nd International Lexical Functional Grammar Conference,
  University of Konstanz. \handout{lfg17_handout.pdf}}

\Reff*{langue2017-talk}{15 June,
  2017}{When the subject honorific
  brings honour to all: the expanding role of pragmatics in Korean
  honorification. Joint work with Yoolim Kim. Presented at the 12th Language at
  the University of Essex Postgraduate Conference (LangUE 2017), Colchester. \slides{kim_findlay-LangUE2017-handout.pdf}}

\Reff*{eacl2017-talk}{4 Apr., 2017}{Multiword
  expressions and lexicalism: the view from LFG. Poster presented at the 13th
  Workshop on Multiword Expressions (MWE 2017), Valencia. \poster{eacl_poster.pdf}}

\Reff*{headlex-talk}{29 July, 2016}{The
  prepositional passive in Lexical Functional Grammar. Presented at the Joint
  2016 Conference on Head-driven Phrase Structure Grammar and Lexical Functional
  Grammar (HeadLex16), Polish Academy of Sciences, Warsaw. \handout{HeadLex_Presentation.pdf}}

\Reff*{igala2014-talk}{5 June, 2014}{Unnatural
  acts lead to unconsummated marriages: discourses of homosexuality in the House
  of Lords revisited. Presented at the 8th International Gender and Language
  Association Conference, Simon Fraser University, Vancouver. \handout{IGALA8Presentation.pdf}}

\subsection*{Invited Talks}

\Reff*{manchester-seminar}{27 Nov.,
  2018}{When the cat you let out of the bag
  has claws: the role of metaphor in understanding idioms. Presented at the
  Linguistics and English Language Research Seminar, University of Manchester. \slides{manchester-lel-seminar.pdf}}

\Reff*{frankfurt-seminar}{25 June, 2018}{How
  (not) to analyse multiword expressions. Presented at the Linguistics
  Oberseminar, Goethe-Universit\"{a}t Frankfurt am Main. \slides{frankfurt-oberseminar.pdf}}

\Reff*{soas-seminar-2017}{31 Jan., 2017}{The pseudopassive: not so phoney after
  all. Presented at the Linguistics Departmental Seminar Series, School of
  Oriental and African Studies (SOAS), London.
  \handout{SOAS_Presentation_p-passive.pdf}}

\newpage

\subsection*{Other Presentations}

\Reff*{fmm2018}{4 Aug., 2018}{Pulling a pretence rabbit out
  of the hat. Joint work with Sascha Bargmann and Manfred Sailer. Presented (by
  Manfred Sailer) at the Workshop on Form-Meaning Mismatches in Natural Language
  (FMM2018), Georg-August-Univers\"{a}t G\"{o}ttingen. \slides{fmm2018.pdf}}

\Reff*{selfg25}{26 May, 2018}{Lexicalised LFG. Presented at the 25th South of
  England LFG Meeting, SOAS, London. \handout{se-lfg25.pdf}}

\Reff*{selfg23}{13 May, 2017}{When the subject honorific brings honour to all:
  the expanding role of pragmatics in Korean honorification. Joint work with
  Yoolim Kim. Presented at the 23rd South of England LFG Meeting, SOAS, London.
  \slides{kim_findlay-SELFG23-handout_slides.pdf}}

\Reff*{selfg22}{4 Feb., 2017}{Multiword expressions and lexicalism. Presented at
  the 22nd South of England LFG Meeting, SOAS, London.
  \handout{SE-LFG_Feb2017-1up.pdf}}

\Reff*{selfg19}{20 Feb., 2016}{Idioms in LFG. Presented at the 19th South of
  England LFG Meeting, SOAS, London. \handout{SELFG_Presentation_idioms.pdf}}

\Reff*{selfg15}{1 Nov., 2014}{Mapping theory without argument structure.
  Presented at the 15th South of England LFG Meeting, SOAS, London.
  \handout{SELFG_Presentation_mapping-theory.pdf}}

\Reff*{mphil-prs}{22 Oct., 2013}{The pseudopassive: a Lexical Functional
  account. Presented at the MPhil/PRS Thesis Workshop, University of Oxford.
  \handout{MPhil_thesis_workshop_presentation.pdf}}

\Reff*{swg-tt2013}{8 May, 2013}{The pseudopassive in LFG: some preliminary
  thoughts. Presented at the Syntax Working Group, University of Oxford.
  \handout{P-passive_SWG.pdf}}

\end{document}
