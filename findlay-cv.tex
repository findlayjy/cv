%==============================================================================%
%                                                                              %
%  file:   findlay-cv.tex                                                      %
%  author: Jamie Findlay <jy.findlay@gmail.com>                                %
%                                                                              %
%  This document heavily inspired by the beautiful CVs of Brian Buccola        %
%  (http://brianbuccola.com/) and Tim Rocktäschel (https://rockt.github.io/)   %
%                                                                              %
%==============================================================================%

%============%
%            %
%  PREAMBLE  %
%            %
%============%

\documentclass[11pt,a4paper]{article}

%===============================%
%  Packages and basic settings  %
%===============================%

\newcommand{\name}{Dr Jamie Y. Findlay}

\usepackage{graphicx}

% Font settings
\usepackage[T1]{fontenc}
\usepackage[utf8]{inputenc}
\usepackage{microtype}
\usepackage[normalem]{ulem}
%
\usepackage[sfdefault,scaled=.85]{FiraSans}
\usepackage{newtxsf}
%
\usepackage{setspace}

% Layout
\usepackage{multicol}
\usepackage{enumitem}
\usepackage{layouts}

% Margins and columns
\usepackage[margin=1.2in,lmargin=2.3cm,rmargin=2.3cm,bmargin=1in,centering,marginparsep=0.8cm]{geometry}

% No indent for new paragraphs
\setlength{\parindent}{0pt}

% Ratios for left and right columns in cvsections and elsewhere
\newcommand{\dateratio}{0.152}
\newcommand{\bodyratio}{0.82}

% Avoid hyphenation across lines
\tolerance=1
\emergencystretch=\maxdimen
\hyphenpenalty=10000
\hbadness=10000

% Some lengths
\newlength{\rulelength}% The length of the rule in section headings
\setlength{\rulelength}{\dateratio\textwidth}
% \addtolength{\rulelength}{11pt}
%
\newlength{\indentlength}% The length of the indent in the \Ref command
\setlength{\indentlength}{29pt}
% \settowidth{\indentlength}{2017}%
% \addtolength{\indentlength}{1em}%

% For manipulating and combining lengths easily
\usepackage{calc}

% Set some colours
\usepackage[svgnames]{xcolor}
% \definecolor{gold}{HTML}{c69c6e}
% \definecolor{green}{HTML}{008000}
% \definecolor{green}{HTML}{39B54A}
\definecolor{hlinkcolor}{rgb}{.0,.2,.4}

% Oxford blue
% \definecolor{oxfordblue}{RGB}{4,30,66}
\definecolor{oxfordblue}{HTML}{002147}

% Oslo colours according to branding guidelines here: https://www.uio.no/om/designmanual/profilelementer/fargepalett/index.html/
% \definecolor{oslored}{HTML}{e2231a} % Don't know where this code is from
\definecolor{oslosealred}{HTML}{B60000}
% \definecolor{oslored}{HTML}{DD0000}
% \definecolor{osloblue}{HTML}{3E31D6}



% Set the colour for the header and header rules
% \colorlet{headercolor}{oxfordblue}
\colorlet{headercolor}{oslosealred!75}



% Different heading command for research experience sections
\newcommand{\REx}[2]{%
\vspace*{0.1\baselineskip}%
% \rule[0.4ex]{\rulelength}{1pt}\hspace*{11pt}%
{\large\textbf{#1}\hfill\textnormal{[#2]}}%
\vspace*{0.5\baselineskip}%
}
%Slightly nicer look but fits less in
% \newcommand{\REx}[2]{%
% \vspace*{0.5\baselineskip}%
% % % \rule[0.4ex]{\rulelength}{1pt}\hspace*{11pt}%
% {\large\textbf{#1}\hfill\textnormal{[#2]}}%
% \vspace*{0.7\baselineskip}%
% }

% Headers, footers, and page styles
\usepackage{titleps}
\newpagestyle{main}[\footnotesize]{%
  \headrule
  \sethead[][][\name]%
          {}{}{\name}
  \setfoot[][\thepage][\scriptsize \today]
          {}{\thepage}{\scriptsize \today}
}
\newpagestyle{first}[\footnotesize]{%
  \sethead{}{}{}
  \setfoot{}{}{\scriptsize \today}
}
\pagestyle{main}

% Packages for CV environment
\usepackage{array}% required for defining newcolumntype with custom vrule
\usepackage{longtable}% normal \tabular environment does not allow page breaks
\setlength{\LTpre}{0pt}% glue before longtable
\addtolength{\LTpost}{0pt}% glue after longtable

% Main body sections
\newcolumntype{L}{>{\raggedleft\footnotesize}p{\dateratio\textwidth}}
\newcolumntype{R}{p{\bodyratio\textwidth}}
%
\newenvironment{cvsection}{%
  \setlength{\extrarowheight}{0.70ex}
  \begin{longtable}[l]{@{} L R @{}}
  %% \begin{longtable}[l]{@{}L@{} @{}R@{}} % Brian's fix of 14/03/21
}{%
  \end{longtable}
}

% Lists for reviews etc.
\newlength{\squish}
\setlength{\squish}{-14.275pt}
% \setlength{\squish}{-8pt}
%% 14pt is the sum of the default values of \topsep (8pt), \partopsep (2pt), and
%% \parsep (4pt). This isn't quite right: the list is too low. But if you use
%% the values these lengths have in this document (9pt, 3pt, and 4.5pt), the
%% list is too high up ... Hence the super-hacky technique.

\newenvironment{reviewlist}
{%
\vspace*{-8pt}%
\begin{itemize}[noitemsep,label={},nosep,left=0pt .. \parindent]%
}
{%
\end{itemize}
}

% \newenvironment{languagelist}
% {\begin{reviewlist}}
% {\end{reviewlist}}

% Allow starred command names
\usepackage{suffix}

% Hyperlinks
\usepackage[breaklinks,hidelinks]{hyperref}
\hypersetup{%
colorlinks=true,%
urlcolor=hlinkcolor,%
linkcolor=hlinkcolor,%
% citecolor = black%
}

\urlstyle{same}


% Enable margin notes everywhere
\usepackage{marginnote}

% Force marginnotes to be on the LHS
\usepackage{etoolbox}
\makeatletter
\patchcmd{\@mn@margintest}{\@tempswafalse}{\@tempswatrue}{}{}
\patchcmd{\@mn@margintest}{\@tempswafalse}{\@tempswatrue}{}{}
\reversemarginpar
\makeatother

% Allow tables to fit to line width
\usepackage{tabularx}

% Icons
\usepackage{fontawesome} % Provides icon commands
\newcommand{\icon}[1]{\raisebox{-.2\dp\strutbox}{#1}} % Aligns icons vertically
\newcommand{\oicon}[1]{\raisebox{-0.5\dp\strutbox}{#1}} % For the ORCID icon


%==========%
%  Macros  %
%==========%

% For references list
\newcounter{RefNo} % Counter for publications

% Just adds a number
\newcommand{\refmark}{%
  \refstepcounter{RefNo}%
  \marginnote{[\theRefNo]}
}

% The \Reff command takes three arguments: a x-ref label, a date, and the bibliographical info
\newcommand\Reff[3]{%
  \refmark%
  \hangindent=\indentlength%
  \label{#1}%
  \textbf{#2}%
  % \hspace*{1em}%
  \hspace{\widthof{2017}+1em-\widthof{#2}}% Set a fixed width for the indent, since e.g. '2020' is wider than '2011'
  #3%
  \vspace*{0.7ex}%
  }

% A starred version for presentations, which have a longer date
\WithSuffix\newcommand\Reff*[3]{%
  \hangindent=\indentlength%
  \refmark%
  \label{#1}%
  \textbf{#2}%
  \hspace*{0.5em}%
  #3%
  \vspace*{0.7ex}%
}

%%%%%%%%%%%%%%%%%%%%%%%%%%%%%%%%%%%%%%%%
%% Commands for hyperlinks to the web %%=========================================================================
%%%%%%%%%%%%%%%%%%%%%%%%%%%%%%%%%%%%%%%%

% Files stored in my webspace -- set the first macro to wherever things live at the moment
\newcommand{\myrepo}{https://jyfindlay.com/linguistics} % github papers repo

% Assuming the following directory structure:
% repo
% └── research
%     ├── drafts
%     ├── papers
%     └── talks
% └── teaching
\newcommand{\myresearch}{research}
\newcommand{\myteaching}{teaching}
\newcommand{\mytalks}{\myresearch/talks}
\newcommand{\mypapers}{\myresearch/papers}
\newcommand{\mydrafts}{\myresearch/drafts}

% Link for files stored in my webspace
\newcommand{\mylink}[2]{\href{\myrepo/#1}{#2}}
% Starred version with underlining
\WithSuffix\newcommand\mylink*[2]{\href{\myrepo/#1}{\uline{#2}}}

% More specific links
% Talks
\newcommand{\mytalkslink}[2]{\mylink{\mytalks/#1}{#2}}
\WithSuffix\newcommand\mytalkslink*[2]{\mylink*{\mytalks/#1}{#2}}
% Papers
\newcommand{\mypaperslink}[2]{\mylink{\mypapers/#1}{#2}}
\WithSuffix\newcommand\mypaperslink*[2]{\mylink*{\myresearch/#1}{#2}}
% Drafts
\newcommand{\mydraftslink}[2]{\mylink{\mydrafts/#1}{#2}}
\WithSuffix\newcommand\mydraftslink*[2]{\mylink*{\mydrafts/#1}{#2}}
% Teaching
\newcommand{\myteachinglink}[2]{\mylink{\myteaching/#1}{#2}}
\WithSuffix\newcommand\myteachinglink*[2]{\mylink*{\myteaching/#1}{#2}}

% Link commands for presentation types
\newcommand{\handout}[1]{\mytalkslink{#1}{\texttt{[handout]}}}
\newcommand{\poster}[1]{\mytalkslink{#1}{\texttt{[poster]}}}
\newcommand{\slides}[1]{\mytalkslink{#1}{\texttt{[slides]}}}

% For links to other URLs
\newcommand{\linkout}[2]{\href{#1}{#2}}
\WithSuffix\newcommand\linkout*[2]{\href{#1}{\uline{#2}}}
% Version with typewriter font and square brackets
% \newcommand{\publinkout}[2]{\href{#1}{\texttt{[#2]}}}

% Command for open access links, which takes a colour argument -- NOT CURRENTLY IN USE
% \newcommand{\oa}[2]{\marginnote{\hspace*{-1em}{\href{#2}{\texttt{[\textcolor{#1}{paper}]}}}}}

% Formatting DOIs
\newcommand{\doi}[1]{DOI:~\href{https://doi.org/#1}{#1}}

%================================================================================================================


% Puts square brackets around label numbers when x-ref'd
\newcommand{\sref}[1]{[\ref{#1}]}

% A note macro
\newcommand{\note}{\emph{Note: }}

% Long dates
\newcommand{\longdate}[1]{\parbox[t]{\dateratio\textwidth}{\raggedleft
#1}}

% Changing font for \LaTeX command because it looks gross in Fira Sans
\WithSuffix\newcommand\LaTeX*{%
  {%
  \fontfamily{cmr}%
  \selectfont\LaTeX%
  }%
}

% Doing the same for \TeX for consistency
\WithSuffix\newcommand\TeX*{%
  {%
  \fontfamily{cmr}%
  \selectfont\TeX%
  }%
}

% Additional info notes for main body entries
\newcommand{\Note}[2]{%
\parbox[t]{\bodyratio\textwidth}{#1\\[-0.25em]{\footnotesize #2}}%
}

% Labels for the left-hand column in main body entries
\newcommand{\Label}[1]{%
\textnormal{#1}%
}

% Subheadings in research experience
\newcommand{\researchsubhead}[1]{%
\textsc{#1}:%
}

% SECTION COMMANDS
\newcommand{\cvheading}[1]{\noindent{{\color{headercolor}\rule[0.4ex]{\rulelength}{2pt}\hspace*{9pt} \Large #1}}\vspace*{0.5\baselineskip}}

\newcommand{\cvsubhead}[1]{\noindent\hspace*{\rulelength}\hspace*{9pt} \textsc{#1}\vspace*{0.25\baselineskip}}

\newcommand{\rulesubhead}[1]{\noindent{\color{headercolor}\rule[0.4ex]{\rulelength}{1pt}\hspace*{9pt} {#1}}\vspace*{0.25\baselineskip}}

\newcommand{\cvsubsubhead}[1]{\noindent\hspace*{\rulelength}\hspace*{9pt} \textit{#1}\vspace*{0.25\baselineskip}}

\newcommand{\pubsubhead}[1]{\noindent\hspace*{\rulelength}\hspace*{9pt} \textsc{#1}\vspace*{0.25\baselineskip}}



% Contact info environment
\newcommand{\ContactInfo}[1]{%
\parbox[c]{\hsize}{\raggedleft\footnotesize\it%
#1}%
}

% Course convenor annotation
\newcommand{\conv}{
% \reversemarginpar
% \marginnote{%
$\mathcal{C}''\hfill%
% }
% \normalmarginpar
}

%========%
%        %
%  BODY  %
%        %
%========%

\frenchspacing

\begin{document}

\thispagestyle{first}

\begin{tabularx}{\linewidth}{@{}lX@{}}
{\Huge\textbf{\name}}%
&\ContactInfo{%
%% Centre for Linguistics and Philology\\
%% Walton St., Oxford, OX1 2HG\\
% United Kingdom\\
\href{mailto:jamie.findlay@iln.uio.no}{{\icon{\faEnvelopeO}\ jamie.findlay@iln.uio.no}}\\
  \href{http://jyfindlay.com}{\icon{\faChain}\ jyfindlay.com}\\
  \href{http://www.linkedin.com/in/findlayjy}{\icon{\faLinkedinSquare}\ findlayjy}\\
  \href{https://github.com/findlayjy}{\icon{\faGithub}\ findlayjy}\\
  \href{https://scholar.google.com/citations?user=Q2v46FwAAAAJ}{\icon{\faGraduationCap}\ Google Scholar profile}\\ \href{https://orcid.org/0000-0002-7700-1525}{\oicon{\includegraphics[height=2ex]{orcid-hirez}} orcid.org/0000-0002-7700-1525}
}
\end{tabularx}

%%==========================================================================%%

\cvheading{Expertise}

\begin{cvsection}
  & Syntax, semantics, computational linguistics, grammatical theory and the
  syntax-semantics interface (especially Lexical Functional Grammar and Glue
  Semantics), pragmatics, (critical)~discourse analysis, language \& gender,
  language \& sexuality.
\end{cvsection}

%%==========================================================================%%

\cvheading{Academic employment}

\begin{cvsection}
  \longdate{01/11/2024--\\present}    & \textbf{Associate Professor (Førsteamanuensis) in Linguistics}, University of Oslo.\\
  \longdate{01/04/2021--\\31/10/2024} & \textbf{Postdoctoral Researcher (Postdoktor) in Computational Semantics}, University of Oslo.\\
  \longdate{01/10/2019--\\31/03/2021} & \textbf{Departmental Lecturer in Syntax}, University of Oxford.\\
  \longdate{01/10/2019--\\ 30/09/2020}& \textbf{Stipendiary Lecturer in Linguistics}, Jesus College, University of Oxford.\\
  \longdate{01/10/2018--\\ 30/09/2019} & \textbf{Stipendiary Lecturer in
    Linguistics}, St Hugh's College, University of Oxford.
\end{cvsection}

%%==========================================================================%%

\cvheading{Education}

\begin{cvsection}
  \longdate{11/10/2015--\\06/05/2019} &
  \Note{\textbf{D.Phil. in Comparative Philology \& General Linguistics.}\\
    University of Oxford.}%
  {\textsc{Thesis title}: \emph{Multiword expressions and the lexicon}  \linkout{https://ora.ox.ac.uk/objects/uuid:502e8ca5-02f7-4be4-8778-cd89364ba670/download_file?file_format=pdf&safe_filename=findlay-thesis-deposit.pdf&type_of_work=Thesis}{\texttt{[pdf]}}.  \\[-0.15em]
    \textsc{Supervisors}: Ash Asudeh, Mary Dalrymple. %[-0.5em]
    \textsc{Examiners}: Stephen Pulman, Tom Wasow. }
  \\
  \longdate{09/10/2012--\\30/06/2014}&
  \Note{\textbf{M.Phil. (with distinction) in General Linguistics \& Comparative Philology.}\\
    University of Oxford.}%
  {\textsc{Thesis title}: \emph{The prepositional passive: a Lexical Functional account} \linkout{https://ora.ox.ac.uk/objects/uuid:86913330-8690-4f22-bc30-3df64f4cf740/download_file?file_format=application\%2Fpdf&safe_filename=Master\%2BDocument.pdf&type_of_work=Thesis}{\texttt{[pdf]}}.\\[-0.15em]
    \textsc{Supervisor}: Mary Dalrymple. }
  \\
  \longdate{01/09/2011--\\06/12/2011} & \Note{Graduate Programme in Linguistics.\\
    McGill University.} {Attended for one semester. GPA 3.92.
  }
  \\
  \longdate{09/10/2007--\\30/06/2011} & \parbox[t]{\bodyratio\textwidth}{\textbf{B.A. (with First Class Honours) in Modern Languages (French) \& Linguistics.}\\
    University of Oxford.}
\end{cvsection}


%%==========================================================================%%

\cvheading{Teaching qualifications}

\begin{cvsection}
  05/12/2017 & \Note{%
    \textbf{SEDA-PDF Supporting Learning award.}%
  }%
  {Mapped to Descriptor 1 of the UK Professional Standards Framework: equivalent to Associate Fellow\\[-0.5em] of the Higher Education Academy. My portfolio submission is available online \myteachinglink*{portfolio-JYF-submission.pdf}{here}. }%
\end{cvsection}

%%==========================================================================%%

\cvheading{Other professional experience}

\begin{cvsection}
    \longdate{01/12/2019--31/03/2021} & \Note{\textbf{Outreach \& Access Officer},
    Faculty of
    Linguistics, Philology \& Phonetics, University of Oxford.}{Responsible for local and national outreach to schools, and organising the \linkout*{https://www.uniq.ox.ac.uk}{UNIQ summer school} in linguistics.}\\
  \longdate{01/07/2012--01/10/2018} & \Note{\textbf{Linguistic
      Consultant}, Freelance.}{Expert consultancy services in and around general linguistics and NLP. This involved data annotation, resource\\[-0.5em]
    mining and compilation, data analysis, and creative copywriting. I worked with researchers at the University\\[-0.5em]
    of Oxford and at Queen Mary University London, as well as with private enterprises such as TheySay Analytics.}\\
  \longdate{01/12/2014--30/09/2015} & \textbf{Part-time Research Assistant},
  Oxford English Dictionary, Oxford University Press.
\end{cvsection}

%%==========================================================================%%
\newpage

\cvheading{Research experience}

\note\ My publications and talks are given on pp.~\pageref{scholarship}ff. This section gives details of two recent projects, with (clickable) cross-references to related publications and talks.\\

\REx{Universal natural language understanding}{semantics, computational
  linguistics}
\begin{cvsection}
  \researchsubhead{Project overview} &%
  This Research Council of Norway-funded project aims to develop and implement a
  rule-based semantic interpretation system, built on the language-agnostic
  Universal Dependencies syntactic framework, and using Glue Semantics as the
  syntax-semantics interface. We are building a highly modular system, where the
  core predicate-argument analysis relies solely on information available from
  the UD parse, and can therefore be applied to any language for which a UD
  parser exists. Additional language-specific resources (such as valency lexica
  or word-sense databases) can then be used in addition if they exist, but are
  not required.
  \\
  \researchsubhead{My role} &%
  I am a post-doctoral researcher on this project, working under the P.I. Dag T.
  T. Haug. I am jointly responsible with the P.I. for developing the rules used
  in translating the syntactic parse into semantic resources, and for managing
  and training annotators who we use to generate gold-standard data from
  existing UD corpora. I also work on developing theoretical insights about the
  syntax-semantics interface from our experience with implementing analyses of a
  wide range of phenomena.
  \\
  \researchsubhead{Outputs} &%
  The project pipeline is presented in \sref{unlu-udw-23}. We have also produced
  a corpus of naturally-ocurring texts with semantic annotations, called
  \textsc{drastic} \sref{dmr23-paper}. Our more theoretical outputs include a
  paper exploring the inadequacies of existing quasi-semantic extensions to
  UD~\sref{depling21}, one developing a new theoretical approach to handling
  scope islands in Glue Semantics and LFG~\sref{lfg22-paper}, and one
  introducing a syntax-semantics interface for Dependency
  Grammar~\sref{semantics-for-dep-gram}.
\end{cvsection}

\REx{Multiword expressions and the lexicon}{syntax, semantics}
\begin{cvsection}
  \researchsubhead{Project overview} &%
  This was my D.Phil. project, funded by the UK Arts and Humanities Research
  Council. My thesis~\sref{thesis} gives an analysis of multiword expressions
  (MWEs), especially idioms. It argues against approaches which conceive of
  idioms like \emph{spill the beans} as being made up of special versions of
  \emph{spill} and \emph{beans}, and in favour of approaches which reaffirm the
  mixed nature of MWEs as partially word-like and partially phrase-like. It then
  develops a concrete example of the latter kind of theory, integrating insights
  from Tree-Adjoining Grammar into the Lexical Functional Grammar architecture.%
  \\
  \researchsubhead{My role} &%
  I was solely responsible for this project from conception to completion, under
  the supervision of my D.Phil. supervisors, Ash Asudeh and Mary Dalrymple. I
  developed the research proposal, disseminated the work through presentations
  and publications, and successfully applied for additional research funding,
  including for a research visit to Germany.
  \\
  \researchsubhead{Outputs} &%
  I have presented this work at peer-reviewed conferences for computational
  linguists \sref{eacl2017-talk}, theoretical syntacticians \sref{lfg2017-talk},
  and for more general linguistics audiences \sref{dgfs2018-talk}. The work has
  also appeared in the proceedings of two of these conferences \sref{lfg2017},
  \sref{eacl2017}. I organised a collaboration with researchers at the
  Goethe-Universit\"{a}t Frankfurt am Main, where I conducted a research visit
  in June 2018. This resulted in a series of presentations and an ongoing
  research collaboration \sref{bfs-idiom-extensions}, \sref{manchester-seminar},
  \sref{creative-power-metaphor-poster}, \sref{europhras2019-talk},
  \sref{fmm2018}. My formal proposals for integrating LFG and TAG appear in a
  chapter in the \textit{Handbook of Lexical Functional Grammar}
  \sref{lfg-tag-chapter}, and the theoretical import of this for LFG (viewed as
  a formalisation of Construction Grammar) is discussed in a paper in the
  \textit{Journal of Language Modelling} \sref{lfg-as-a-cxg}.
\end{cvsection}

% %%==========================================================================%%
\newpage

\cvheading{Teaching}

\begin{cvsection}
  &
  At the undergraduate level, I have created and (co-)led courses in \textbf{general linguistics}, \textbf{syntax}, \textbf{semantics \& pragmatics}, \textbf{sociolinguistics}, \textbf{psycholinguistics}, and \textbf{historical linguistics}.\\
  & At the graduate level, I have created and led courses in \textbf{syntax} and \textbf{formal approaches to sociolinguistic variation}, as well as TA-ing for courses in \textbf{syntax}, \textbf{phonology}, and \textbf{formal foundations of linguistics} (an introduction to set theory, logic, and formal language theory for linguists).\\
  & My teaching experience covers small group and tutorial settings, larger classes, and lectures, and includes both curriculum development and the setting \& marking of exams.\\
  & Qualifications: I hold a \textbf{SEDA-PDF Supporting Learning award} (see above for details).\\
  & In this section, years refer to the academic calendar (e.g. 2017 = the
  2017/18 academic year).
\end{cvsection}

\rulesubhead{University of Oslo}

\cvsubhead{Graduate level}

\cvsubsubhead{As primary instructor}

\begin{cvsection}
  2023 & \Note{\textbf{Formal approaches to sociolinguistic
      variation} (seminars -- group size: 12)}{[LING4130: Linguistics Specialization on variation]}
\end{cvsection}

\cvsubsubhead{As guest lecturer}

\begin{cvsection}
  2024 & \Note{\textbf{Linguistic method} (lecture -- group size: 20)}{[LING4140: Språkvitenskapelig metode] -- 1 lecture on `Modelling in the language sciences'}
\end{cvsection}

\cvsubhead{Undergraduate level}

\cvsubsubhead{As primary instructor}

\begin{cvsection}
  2025 & \Note{\textbf{Semantics and pragmatics 2} (lectures -- group size: ~20)}{[LING2100: Semantikk og pragmatikk 2]}
\end{cvsection}

\cvsubsubhead{As joint primary instructor}

\begin{cvsection}
  2022 & \Note{\textbf{Language change} (lectures -- group size:
    \textasciitilde{}20).}{[LING2108: Språkendring]}\\
  2021, 2023 & \Note{\textbf{Psycholinguistics and sociolinguistics} (lectures
    -- group size: \textasciitilde{}20).}{[LING1113: Psykolingvistikk og
    sosiolingvistikk]}
\end{cvsection}

%---------------------------------------

\rulesubhead{University of Oxford}

\cvsubhead{Graduate level}

\cvsubsubhead{As primary instructor}
\begin{cvsection}
    2019--2020        & \textbf{Contemporary syntactic theory} (lectures and classes -- group size: \textasciitilde{}12).\\
    2019--2020        & \textbf{Introduction to syntax} (lectures -- group size: \textasciitilde{}25).
\end{cvsection}

\cvsubsubhead{As TA}
\begin{cvsection}
  2017--2018 & \Note{ \textbf{Formal foundations of linguistics} (classes --
    group size: \textasciitilde{}25).}
  {An introduction to set theory, logic, and formal language theory for linguists.}\\
  2015--2016 & \textbf{Introduction to syntax} (classes -- group
  size: \textasciitilde{}25).\\
  2014 & \textbf{Introduction to phonology} (classes -- group size: 12).
\end{cvsection}

\newpage

\cvsubhead{Advanced undergraduate level}

\begin{cvsection}
    {\mbox{\llap{2014--2016, 2019--2020}}}
                & \Note{\textbf{Syntax} (tutorials -- group size: 1--3).}{[FHS Paper XII/B2]}\\
  2018--2019  & \Note{\textbf{General linguistics} (tutorials -- group size: 2--3).}
              {[FHS Paper XIII/A] -- A survey module which covers the development of
                linguistic theory since the 19th century.}\\
    2017--2019
                & \Note{\textbf{Sociolinguistics} (tutorials -- group size: 1--3).}{[FHS Paper XII/B5]}\\
    2018  & \Note{\textbf{Psycholinguistics} (tutorials -- group size: 1--3).}{[FHS Paper XII/B4]}\\
    2014--2017  & \Note{\textbf{Semantics \& pragmatics} (tutorials -- group size: 1--3).}{[FHS Paper XII/B3]}
  \end{cvsection}

\cvsubhead{First-year undergraduate level}
\begin{cvsection}
    2014, 2018--2020
                & \Note{%
                \textbf{Grammatical analysis} (classes -- group size: 3--6).}
                {[Prelims Paper X] -- An introduction to both syntax and morphology.}\\
    2014, 2018--2019
                &	\Note{%
                \textbf{General linguistics} (tutorials -- group size: 2--3).}
                {[Prelims Paper VIII] -- An introductory course which covers topics in general linguistics proper, along with \\[-0.5em] psycholinguistics, semantics \& pragmatics, sociolinguistics, and historical  linguistics.}\\
    2018        & \Note{\textbf{Psycholinguistics} (lectures -- group size: \textasciitilde{}50).}{For Prelims Paper VIII: General linguistics.}\\
    2017        & \Note{\textbf{Semantics \& pragmatics} (lectures -- group size: \textasciitilde{}50).}{For Prelims Paper VIII: General linguistics.}
\end{cvsection}

\cvsubhead{Irregular classes}
\begin{cvsection}
    2016, 2018	& \Note{%
                  \textbf{Language and identity: the view from linguistics} (class -- group size: \textasciitilde{}6).}
                  {An introduction to sociolinguistics for first year English language and literature undergraduates.\\[-0.25em]
                  Handout available \myteachinglink*{socio_presentation.pdf}{here}.}\\
    2017        & \textbf{Corpus methods in sociolinguistics} (graduate class -- group size: 3).\\
    2016--2017, 2019 & \Note{%
                  \textbf{\LaTeX*\ workshops}: `\LaTeX*\ basics' and `Going further with \LaTeX*'.}
                  {Two workshops to introduce graduate students in linguistics to \LaTeX*.\\[-0.25em]
                  Handouts available \myteachinglink*{latex-basics.pdf}{here} and \myteachinglink*{latex-going-further.pdf}{here}.}
\end{cvsection}

% %%==========================================================================%%
% \newpage
\cvheading{Supervision}

\rulesubhead{University of Oslo}

\cvsubhead{Undergraduate supervision}

\begin{cvsection}
  2024 & Lene Marie Olsen. B.A. thesis (guest supervisor for LING3090: B.A.
  thesis seminar). Title:~\textit{Idiomatic flexibility: an analysis of the
    idiom ``a hill to die on''}.%
\end{cvsection}

%---------------------------------------
\rulesubhead{University of Oxford}

\cvsubhead{Graduate supervision}

\begin{cvsection}
  2021 & Robert Flick. Extended essay in syntax. Title:~\textit{Distinguishing
    middles and passives in Hungarian: a numerical information structure
    approach}. M.St. in Linguistics, Philology and Phonetics. \\

  2021 & Kamran Sharifi. Extended essay in syntax. Title:~\textit{``Crazy
    syntax'': examining the role of the synchronic/diachronic distinction in
    syntactic theory}. M.St. in Linguistics, Philology and Phonetics.\\

  2021 & Tianxin Tu. Extended essay in syntax. Title:~\textit{The Mandarin
    Chinese \emph{ba} construction revisited: an LFG account}. M.St. in Linguistics, Philology and Phonetics.\\

  2021 & Xiulin Yang. Extended essay in syntax. Title:~\textit{The ambiguous
    binding of \emph{ziji}: an LFG analysis}. M.St. in Linguistics, Philology and Phonetics.\\

  2021 & Tommy Zhang. Extended essay in syntax. Title:~\textit{The concept of
    Function-Argument Biuniqueness in LFG \& Glue and additional evidence
    from reflexives and complex predicates}. M.St. in Linguistics, Philology and Phonetics.\\

  2020 & Chenyu Fang. Extended essay in syntax. Title:~\textit{An LFG analysis
    of anaphoric reflexive \emph{(ta)-ziji} in Mandarin Chinese}. M.St. in
  Linguistics, Philology and Phonetics.
\end{cvsection}

\cvsubhead{Undergraduate supervision}

\begin{cvsection}
  2020--2021 & Caitlyn McDermott. Linguistic project (thesis with a large data
  collection\slash analysis component). Title: \textit{What are they talking about? A discourse analysis of US press coverage of mass shootings}. B.A. in Psychology \& Linguistics.\\
  %
\end{cvsection}

%% ==========================================================================%%

% \cvheading{Mobility}

% \begin{cvsection}
%   June 2018 & \textbf{Visiting researcher}, Goethe-Universit\"{a}t Frankfurt am Main, Department of English and American Studies.
% \end{cvsection}

%%==========================================================================%%

\cvheading{Grants and awards}

\begin{cvsection}

  June--Dec. 2023 & \Note{\textbf{Support for Researcher Mobility}, Research Council of Norway.}{Value: NOK 671,000. Funded a research stay at the Goethe-Universit\"{a}t Frankfurt am Main, including a concomitant \\[-0.5em] extension to my postdoctoral contract.}\\

  May 2018   & \Note{\textbf{Scatcherd European Scholarship}, University of Oxford.}{Value: £1,000. Awarded, along with the following award, for a research visit to the Goethe-Universit\"{a}t Frankfurt am\\[-0.5em] Main.}\\
  %
  Mar. 2018   & \Note{\textbf{Santander Academic Travel Award}, University of Oxford.}{Value: £1,000.}\\
  %
  \longdate{%
    Oct. 2015--\\June 2017}
  % 2015--2017
  & \Note{\textbf{Professor Paul Slack Scholarship}, Linacre College, University of Oxford.}
  {Total value: \textasciitilde{}£14,000 (fees for first 2 years of D.Phil.).}\\
  %
  \longdate{%
    Oct. 2015--\\June 2017}
  % 2015--2017
  & \Note{\textbf{AHRC Studentship}, Faculty of Linguistics, Philology \& Phonetics, University of Oxford.}
  {Total value: £28,353 (living costs for first 2 years of D.Phil.).}\\
  %
  July 2014 & \Note{\textbf{George Wolf Prize}, Faculty of Linguistics, Philology \& Phonetics, University of Oxford.}{Awarded for best overall performance in the M.Phil. or M.St. degree.}\\
  %
  June 2014 & \textbf{Graduate student essay prize}, International Gender and Language Association.\\
  %
  \longdate{%
    Oct. 2012--\\June 2014}
  % 2012--2014
  & \Note{\textbf{AHRC Studentship}, Faculty of Linguistics, Philology \& Phonetics, University of Oxford.}
  {Total value: \textasciitilde{}£38,000 (fees and living costs for M.Phil.).}\\
  %
  \longdate{Sept. 2011--\\June 2013}
  & \Note{\textbf{McCall McBain Fellowship}, McGill University.}
  {Total value: \textasciitilde{}CAN\$62,000 (fees and living costs for 2-year MA degree); partially declined.}\\
  %
  \longdate{%
  Oct. 2008--\\June 2011}
  % 2008--2011
  & \Note{\textbf{Casberd Scholarship}, St John's College, University of Oxford.}
  {Awarded for achieving a distinction in first year examinations. Total value: \textasciitilde{}£900.}
\end{cvsection}

% %%==========================================================================%%

\cvheading{Commissions of trust}

\cvsubhead{External}
\begin{cvsection}
  \Label{Editor} & \textit{Proceedings of the LFG Conference}~(2021--present).\\
  \Label{Chair} & Executive Committee, International Lexical Functional Grammar Association (ILFGA) (Sept.~2023--Aug.~2024).\\ %Aug.~2024).\\
  \Label{Committee member} & Executive Committee, ILFGA (Sept.~2021--Aug.~2024).\\
\Label{Grant reviewer} &
                    Narodowe Centrum Nauki (Polish National Science Centre)~(2019).\\
\Label{Manuscript reviewer} &
                \begin{reviewlist}
                \item \textit{Corpora and Discourse Studies}~(2018),
                \item \textit{Journal of Language and Sexuality}~(2019),
                \item \textit{Journal of Linguistics}~(2015, 2017, 2019, 2024),
                \item Language Science Press~(two chapters: 2020, 2022),
                \item \textit{Poznan Studies in Contemporary Linguistics}~(2019),
                \item \textit{Proceedings of the LFG Conference}~(2018--present),
                \item  \textit{Syntax}~(2020).
                \end{reviewlist}
                 \\[\squish]
\Label{Abstract reviewer}   &
                \begin{reviewlist}
                \item \textit{Conference of the Student Organisation of Linguistics in Europe (ConSOLE)}~(2018),
                \item \textit{International Lexical Functional Grammar Conference} (2019--present).
                \end{reviewlist}
                \\[\squish]
\Label{Marker}      & UK Linguistics Olympiad~(2014--present).\\
\Label{Copy editor} & \textit{Journal of Language Modelling} (2017--2021).
\end{cvsection}
%
\vspace{0.2\squish}
%
\cvsubhead{Institutional}
\begin{cvsection}
  Dec. 2019--Mar. 2021 & \textbf{Outreach \& Access Officer}, Faculty of
  Linguistics, Philology \& Phonetics, University of Oxford.\\
  Oct. 2019--Mar. 2021 & \Note{\textbf{Committee member}, Athena Swan Self Assessment Team, Faculty of Linguistics, Philology \& Phonetics, University of Oxford.}{Athena Swan is a gender equality certification for higher education institutes in the UK.}\\
  Oct. 2018--Mar. 2021 & \textbf{Committee member}, Undergraduate Studies Committee, Faculty of Linguistics, Philology \& Phonetics, University of Oxford.\\
  Oct. 2019--Sept. 2020 & \Note{\textbf{Personal tutor and Linguistics Organising Tutor}, Jesus College.}{Providing pastoral support and coordinating teaching for undergraduate linguistics students.}\\
  \longdate{Jan.--Mar. 2018, \\Jan.--June 2020} & \textbf{Convenor}, Syntax Working Group, University of Oxford.\\
  Dec. 2015--Dec. 2019 & \textbf{Admissions interviewing} for undergraduate degrees involving linguistics, University of Oxford.\\
  Oct. 2018--Sept. 2019  & \textbf{Personal tutor and Linguistics Organising Tutor}, St Hugh's College.\\
  Oct. 2017--June 2018 & \textbf{Co-founder and co-convenor}, D.Phil. workshop,
  Faculty of Linguistics, Philology \& Phonetics, University of Oxford.
    % 2008--2009  & President, Oxford University Linguistics Society.\\
    % 2007--2009  & Student representative, Faculty of Linguistics, Philology \&
                % Phonetics, University of Oxford.\\
\end{cvsection}


%%==========================================================================%%
% \newpage
\cvheading{Languages}

\begin{cvsection}
  \Label{Natural} & English~(mother tongue), French~(fluent), Norwegian~(intermediate),
  Latin~(intermediate), German~(beginner).
  \\
  \Label{Programming}  &     Bash/Zsh, Python, Emacs Lisp.\\
  \Label{Markup} & \LaTeX*\slash Bib\TeX*, HTML\slash CSS.
\end{cvsection}


%%==========================================================================%%
% \newpage
\cvheading{Professional Memberships}

\begin{cvsection}
  Since 2017 & Association for Computational Linguistics (ACL).\\
  Since 2014 & International Lexical Functional Grammar Association (ILFGA).\\
  Since 2012 & Canadian Linguistic Association (CLA\slash ACL).\\
  Since 2011 & Linguistics Association of Great Britain (LAGB).\\
  Since 2011 & International Gender and Language Association (IGALA).\\
\end{cvsection}

%%==========================================================================%%

\newpage
\newcounter{dummy}
\refstepcounter{dummy} % This makes the hyperlinks from above jump to the correct page


\cvheading{Scholarship} \label{scholarship}

Clicking on a title will take you to an openly accessible version of that item.
For conference presentations, I provide a link to the slides, handout, or
poster, as appropriate.\medskip

Number of citations is given per Google Scholar, after self-citations are
excluded.

\subsection*{Theses}

\Reff{thesis}{2019}{Jamie Y. Findlay.
  \linkout{https://ora.ox.ac.uk/objects/uuid:502e8ca5-02f7-4be4-8778-cd89364ba670/download_file?file_format=pdf&safe_filename=findlay-thesis-deposit.pdf&type_of_work=Thesis}{\textit{Multiword
      expressions and the lexicon.}} D.Phil. thesis, University of Oxford.
  \hfill[5 citations]}

\Reff{mphil-thesis}{2014}{Jamie Y. Findlay.
  \linkout{https://ora.ox.ac.uk/objects/uuid:86913330-8690-4f22-bc30-3df64f4cf740/download_file?file_format=application\%2Fpdf&safe_filename=Master\%2BDocument.pdf&type_of_work=Thesis}{\textit{The
      prepositional passive: a Lexical Functional account.}} M.Phil. thesis,
  University of Oxford.\\\null\hfill{[6 citations]}}

\vspace*{-1em}
\subsection*{In Progress}

\Reff*{ud-negation-paper}{In prep.}{Jamie Y. Findlay and Dag T. T. Haug.
  Negation in Universal Dependencies.}

\Reff*{bfs-idiom-extensions}{In prep.}{Sascha Bargmann, Jamie Y. Findlay,
  and Manfred Sailer. Defrosting frozen metaphors: on the extensibility and
  manipulability of idioms. To be submitted to \textit{Glossa}.}

\Reff*{honorifics-data-paper}{In prep.}{Yoolim Kim, Marc Allassonni\`ere-Tang,
  and Jamie Y. Findlay. Finding the target of the `subject' honorific marker in
  Korean. To be submitted to \textit{Glossa Psycholinguistics}.}

\subsection*{Journal Articles and Book Chapters}

\Reff{lfg-as-a-cxg}{2023}{Jamie Y. Findlay.
  \linkout{https://jlm.ipipan.waw.pl/index.php/JLM/article/view/338/279}{Lexical
    Functional Grammar as a Construction Grammar}. \textit{Journal of Language
    Modelling} 11(2), 197--266. \doi{10.15398/jlm.v11i2.338}.}

\Reff{lfg-tag-chapter}{2023}{Jamie Y. Findlay.
  \linkout{https://docs.google.com/gview?url=https://zenodo.org/records/10186054/files/312-dalrymple-2023-43.pdf}{LFG
    and Tree-Adjoining Grammar}. In Mary Dalrymple (ed.), \emph{Handbook of
    Lexical Functional Grammar} (Empirically Oriented Theoretical Morphology and
  Syntax 13), 2069--2125. Language Science Press.
  \doi{10.5281/zenodo.10186054}. \hfill[1 citation]}

\Reff{arg-str-handbook-chapter}{2023}{Jamie Y. Findlay, Roxanne Taylor, and Anna
  Kibort. \linkout{https://docs.google.com/gview?url=https://zenodo.org/records/10185966/files/312-dalrymple-2023-16.pdf}{Argument
    structure and Mapping Theory}. In Mary Dalrymple (ed.), \emph{Handbook of
    Lexical Functional Grammar} (Empirically Oriented Theoretical Morphology and
  Syntax 13), 699--778. Language Science Press. \doi{10.5281/zenodo.10185966}.
  \hfill[13 citations]}

\Reff{meaning-in-lfg-chapter}{2021}{Jamie Y. Findlay.
  \mypaperslink{findlay-2021-meaning_in_lfg.pdf}{Meaning in LFG}. \raggedright
  In I Wayan Arka, Ash Asudeh, and Tracy Holloway King (eds.), \emph{Modular
    design of grammar: linguistics on the edge}, 340--374. Oxford University
  Press. \doi{10.1093/oso/9780192844842.003.0020}. \hfill[2 citations]}

\Reff{lfg-chapter}{2019}{Mary Dalrymple and Jamie Y. Findlay.
  \mypaperslink{dalrymple_findlay-2019-lexical_functional_grammar.pdf}{Lexical
    Functional Grammar}. In Andr\'{a}s Kert\'{e}sz, Edith Moravcsik, and Csilla
  R\'{a}kosi (eds.), \emph{Current approaches to syntax: a comparative
    handbook}, 123--154. De Gruyter Mouton. \doi{10.1515/9783110540253-005}.
  \hfill[8 citations]}

\Reff{house-of-lords-paper}{2017}{Jamie Y. Findlay.
  \mypaperslink{house_of_lords.pdf}{Unnatural acts lead to unconsummated
    marriages: discourses of homosexuality within the {House of Lords} debate on
    same-sex marriage}. \emph{Journal of Language and Sexuality} 6(1), 30--60.
  \doi{10.1075/jls.6.1.02fin}. \hfill[22 citations]}

\Reff{arg-str-paper}{2016}{Jamie Y. Findlay.
  \linkout{https://jlm.ipipan.waw.pl/index.php/JLM/article/view/171/148}{Mapping
    theory without argument structure}. \emph{Journal of Language Modelling}
  4(2), 293--338. \doi{10.15398/jlm.v4i2.171}. \hfill[58 citations]}

\Reff{bmj-paper}{2013}{Janak Bechar, Jamie Y. Findlay, and Joseph Hardwicke.
  \mypaperslink{medical_terminology.pdf}{Understanding medical words of
    Greek and Latin origin}. \emph{Student BMJ} 21. \doi{10.1136/sbmj.f7394}.}

% \newpage
\subsection*{Peer-reviewed Conference Proceedings Papers}

\Reff{dmr23-paper}{2023}{Dag T. T. Haug, Jamie Y. Findlay, and Ahmet
  Y\i{}ld\i{}r\i{}m. \linkout{https://aclanthology.org/2023.dmr-1.9/}{The long
    and the short of it: \textsc{drastic}, a semantically annotated dataset
    containing sentences of more natural length.} In \emph{Proceedings of the
    Fourth International Workshop on Designing Meaning Representations (DMR
    2023)}, 89--98. Association for Computational Linguistics. \hfill{[1
    citation]}}

\Reff{unlu-udw-23}{2023}{Jamie Y. Findlay, Saeedeh Salimifar, Ahmet
  Y\i{}ld\i{}r\i{}m, and Dag T. T. Haug.
  \linkout{https://aclanthology.org/2023.udw-1.6/}{Rule-based semantic
    interpretation for Universal Dependencies}. In \textit{Proceedings of the
    Sixth Workshop on Universal Dependencies (UDW, GURT\slash SyntaxFest 2023)},
  47--57. Association for Computational Linguistics. \hfill[2 citations]}

\Reff{semantics-for-dep-gram}{2023}{Dag T. T. Haug and Jamie Y. Findlay.
  \linkout{https://aclanthology.org/2023.depling-1.3/}{Formal semantics for Dependency Grammar}.
  In \textit{Proceedings of the Seventh International Conference on
    Dependency Linguistics (Depling, GURT\slash SyntaxFest 2023)}, 22--31. Association for Computational
  Linguistics. \hfill[1 citation]}

\Reff{lfg22-paper}{2022}{Jamie Y. Findlay and Dag T. T. Haug.
  \href{https://ojs.ub.uni-konstanz.de/lfg/index.php/main/article/view/18/25}{Managing
    scope ambiguities in Glue via multistage proving}. In Miriam Butt, Jamie Y.
  Findlay, and Ida Toivonen (eds.), \emph{Proceedings of the LFG'22 Conference},
  143--163. Stanford, CA: CSLI Publications. \hfill[1 citation]}

\Reff{depling21}{2021}{Jamie Y. Findlay and Dag T. T. Haug.
  \href{https://aclanthology.org/2021.depling-1.3/}{How useful are Enhanced
    Universal Dependencies for semantic interpretation?} In \textit{Proceedings of
    the Sixth International Conference on Dependency Linguistics (Depling,
    SyntaxFest 2021)}, 22--34. Association for Computational Linguistics. \hfill[3 citations]}

\Reff{lfg2020}{2020}{Jamie Y. Findlay.
 \href{http://web.stanford.edu/group/cslipublications/cslipublications/LFG/LFG-2020/lfg2020-findlay.pdf}{Mapping
    Theory and the anatomy of a verbal lexical entry}. In Miriam Butt and Ida
  Toivonen (eds.), \emph{Proceedings of the LFG'20 Conference}, 127--147.
  Stanford, CA: CSLI Publications. \hfill[13 citations]}

\Reff{lfg2017}{2017}{Jamie Y. Findlay.
  \href{http://web.stanford.edu/group/cslipublications/cslipublications/LFG/LFG-2017/lfg2017-findlay.pdf}{Multiword
    expressions and lexicalism}. In Miriam Butt and Tracy Holloway King (eds.),
  \emph{Proceedings of the LFG'17 Conference}, 209--229. Stanford, CA: CSLI
  Publications. \hfill[3 citations]}

\Reff{eacl2017}{2017}{Jamie Y. Findlay.
  \linkout{https://www.aclweb.org/anthology/W17-1709.pdf}{Multiword expressions
    and lexicalism: the view from LFG}. In \emph{Proceedings of the 13th
    Workshop on Multiword Expressions (MWE 2017)}, 73--79. Association for
  Computational Linguistics.
  DOI:~\href{https://doi.org/10.18653/v1/W17-1709}{10.18653/v1/W17-1709}.
  \hfill[2 citations]}

\Reff{headlex}{2016}{Jamie Y. Findlay.
 \linkout{https://proceedings.hpsg.xyz/article/view/533}{The
    prepositional passive in Lexical Functional Grammar}. In Doug Arnold, Miriam
  Butt, Berthold Crysmann, Tracy Holloway King, and Stefan M\"uller (eds.),
  \emph{Proceedings of the Joint 2016 Conference on Head-driven Phrase Structure
    Grammar and Lexical Functional Grammar}, 255--275. Stanford, CA: CSLI
  Publications. \hfill[12 citations]}

\subsection*{Unreviewed Conference Proceedings Papers}

\Reff{lsa23-proceedings}{2023}{Yoolim Kim and Jamie Y. Findlay.
  \linkout{https://journals.linguisticsociety.org/proceedings/index.php/PLSA/article/view/5534/5188}{On
    the `subject' honorific \textit{-si-} in Korean}. \textit{Proceedings of the
    Linguistic Society of America (PLSA)} 8(1), 5534.
  \doi{10.3765/plsa.v8i1.5534}}

\subsection*{Invited Talks}

\Reff*{konstanz-colloquium-2023}{16 Nov., 2023}{The continued relevance of
  rule-based semantic parsing in the era of deep learning: an introduction to
  the Universal Natural Language Understanding project. Presented at the
  Department of Linguistics Department and Research Colloquium (Fachbereichs-
  und Forschungsgruppenkolloquium), Universit\"{a}t Konstanz.
  \slides{konstanz-seminar-2023-handout.pdf}}

\Reff*{frankfurt-obserseminar-parsing}{13 Nov., 2023}{Parsing, fast and slow: a
  two-system approach to some topics in linguistics. Presented at the
  Obserseminar Syntax and Semantics, Goethe-Universit\"{a}t Frankfurt am Main. \slides{2023-11-13-gu-oberseminar-slides-findlay-handout.pdf}}

\Reff*{frankfurt-oberseminar-2023}{12 June, 2023}{The continued potential of
  rule-based semantic parsing in the era of deep learning. Presented at the
  Oberseminar Syntax and Semantics, Goethe-Universit\"{a}t Frankfurt am Main.
  \slides{frankfurt-oberseminar-2023.pdf}}

\Reff*{rochester-seminar}{1 Dec., 2022}{Lexical Functional Grammar as a
  Construction Grammar. Presented online at the joint Rochester-Carleton
  L$_{\textnormal{R}}$FG Research Lab, University of Rochester.
  \slides{lrfg-lab-talk.pdf}}

\Reff*{manchester-seminar}{27 Nov., 2018}{When the cat you let out of the bag
  has claws: the role of metaphor in understanding idioms. Presented at the
  Linguistics and English Language Research Seminar, University of Manchester.
  \slides{manchester-lel-seminar.pdf}}

\newpage

\Reff*{frankfurt-oberseminar-2018}{25 June, 2018}{How (not) to analyse multiword
  expressions. Presented at the Oberseminar Syntax and Semantics,
  Goethe-Universit\"{a}t Frankfurt am Main. \slides{frankfurt-oberseminar-2018.pdf}}

\Reff*{soas-seminar-2017}{31 Jan., 2017}{The pseudopassive: not so phoney after
  all. Presented at the Linguistics Departmental Seminar Series, School of
  Oriental and African Studies (SOAS), London.
  \handout{SOAS_Presentation_p-passive.pdf}}

\subsection*{Peer-reviewed Presentations}

\Reff*{lfg23-talk}{24 July, 2023}{Lexical integrity is the norm: or, ``violating
  lexical integrity in a specific and limited way''. Presented at the workshop
  `Revisiting lexical integrity' during the 28th International
  Lexical-Functional Grammar Conference, University of Rochester.
  \slides{LFG23-slides.pdf}}

\Reff*{dmr23-talk}{20 June, 2023}{The long and the short of it: a new dataset of
  semantically annotated sentences of more natural length. Joint work with Dag
  T. T. Haug and Ahmet Y\i{}ld\i{}r\i{}m. Presented at the Fourth International
  Workshop on Designing Meaning Representations (DMR 2023), Nancy, France.
  \slides{dmr23-slides.pdf}}

\Reff*{unlu-udw-23-talk}{11 Mar., 2023}{Rule-based semantic interpretation for
  Universal Dependencies. Joint work with Saeedeh Salimifar, Ahmet
  Y\i{}ld\i{}r\i{}m and Dag T. T. Haug. Presented at the Sixth Workshop on
  Universal Dependencies (GURT\slash SyntaxFest 2023), Georgetown University,
  Washington, D.C. \slides{udw2023-slides.pdf}}

\Reff*{semantics-for-dep-gram-talk}{10 Mar., 2023}{Semantics for Dependency
  Grammar. Joint work with Dag T. T. Haug. Presented (by Dag Haug) at the
  Seventh International Conference on Dependency Linguistics (GURT\slash
  SyntaxFest 2023), Georgetown University, Washington, D.C.
  \slides{depling23-slides.pdf}}

\Reff*{lsa23-talk}{6 Jan., 2023}{The `subject' honorific \textit{-si-} in
  Korean. Joint work with Yoolim Kim. Presented (by Yoolim Kim) at the 97th Annual Meeting
  of the Linguistics Society of America, Denver, Colorado.
  \slides{kim-findlay-lsa23-slides.pdf}}

\Reff*{lfg22-talk}{12 July, 2022}{Managing scope ambiguities in Glue via
  multistage proving. Joint work with Dag T. T. Haug. Presented at the 27th
  International Lexical Functional Grammar Conference, University of Groningen.
  \slides{findlay-haug-lfg22-slides-handout.pdf}}

\Reff*{depling21-talk}{22 Mar., 2022}{How useful are Enhanced Universal
  Dependencies for semantic interpretation? Joint work with Dag T. T. Haug.
  Presented online at the Sixth International Conference on Dependency
  Linguistics (Depling 2021), Sofia. \slides{DepLing22_slides-handout.pdf}}

\Reff*{iccg11-talk}{20 Aug., 2021}{Constructions in LFG. Presented online at the
  Workshop on Constructional Approaches in Formal Grammar at the 11th
  International Conference on Construction Grammar (ICCG11), University of
  Antwerp. \slides{iccg-handout-findlay.pdf}}

\Reff*{lfg2020-talk}{25 June, 2020}{Lexical Mapping Theory and the anatomy of a
  lexical entry. Presented online at the 25th International Lexical Functional
  Grammar Conference, University of Oslo. \slides{jfindlay-lfg2020-slides.pdf}}

\Reff*{creative-power-metaphor-poster}{30 Mar., 2019}{Extended idioms as
  metaphors. Joint work with Sascha Bargmann and Manfred Sailer. Poster
  presented at the Conference on the Creative Power of Metaphor, Worcester
  College, University of Oxford. \poster{FBS-poster.pdf}}

\Reff*{europhras2019-talk}{24 Jan., 2019}{Why the butterflies in your stomach
  can have big wings: combining formal and cognitive theories to explain
  productive extensions of idioms. Joint work with Sascha Bargmann and Manfred
  Sailer. Presented at the International Conference of the Europ\"{a}ische
  Gesellschaft f\"{u}r Phraseologie (EUROPHRAS 2019): Productive Patterns in
  Phraseology, Santiago de Compostela, Spain. \slides{europhras19.pdf}}

\Reff*{lfg2018-talk}{18 July, 2018}{Lexical and grammatical meaning: attributive
  adjectives in French. Joint work with Hannah Senior. Presented at the 23rd
  International Lexical Functional Grammar Conference, University of Vienna.
  \slides{findlay-senior-lfg2018-slides.pdf}}

\Reff*{dgfs2018-talk}{8 Mar., 2018}{Conspiracy theories: the problem with
  lexical approaches to idioms. Presented at the Workshop on One-to-Many
  Relations in Morphology, Syntax and Semantics at the 40th Annual Meeting of
  the Deutsche Gesellschaft f\"{u}r Sprachwissenschaft (DGfS 2018),
  Universit\"{a}t Stuttgart. \slides{findlay-dgfs-slides.pdf}}

\newpage

\Reff*{lagb2017-talk}{6 Sept., 2017}{The logic of Korean honorification. Joint
  work with Yoolim Kim. Presented at the 2017 Annual Meeting of the Linguistics
  Association of Great Britain (LAGB 2017), University of Kent, Canterbury.
  \slides{kim-findlay-lagb.pdf}}

\Reff*{lfg2017-talk}{27 July, 2017}{Multiword expressions and lexicalism.
  Presented at the 22nd International Lexical Functional Grammar Conference,
  University of Konstanz. \handout{lfg17_handout.pdf}}

\Reff*{langue2017-talk}{15 June, 2017}{When the subject honorific brings honour
  to all: the expanding role of pragmatics in Korean honorification. Joint work
  with Yoolim Kim. Presented at the 12th Language at the University of Essex
  Postgraduate Conference (LangUE 2017), Colchester.
  \slides{kim_findlay-LangUE2017-handout.pdf}}

\Reff*{eacl2017-talk}{4 Apr., 2017}{Multiword expressions and lexicalism: the
  view from LFG. Poster presented at the 13th Workshop on Multiword Expressions
  (MWE 2017), Valencia. \poster{eacl_poster.pdf}}

\Reff*{headlex-talk}{29 July, 2016}{The prepositional passive in Lexical
  Functional Grammar. Presented at the Joint 2016 Conference on Head-driven
  Phrase Structure Grammar and Lexical Functional Grammar (HeadLex16), Polish
  Academy of Sciences, Warsaw. \handout{HeadLex_Presentation.pdf}}

\Reff*{igala2014-talk}{5 June, 2014}{Unnatural acts lead to unconsummated
  marriages: discourses of homosexuality in the House of Lords revisited.
  Presented at the 8th International Gender and Language Association Conference,
  Simon Fraser University, Vancouver. \handout{IGALA8Presentation.pdf}}


\subsection*{Other Presentations}

\Reff*{iln-seminar-oct-23}{6 Oct., 2023}{Parsing, fast and slow: a two-system
  approach to some topics in linguistics. Presented at the Linguistics
  Departmental Seminar (Spr\aa kvitenskapelig Instituttseminar), Department of
  Linguistics and Scandinavian Studies, University of Oslo.
  \slides{2023-10-06-iln-departmental-seminar-slides.pdf}}

\Reff*{selfg32}{21 May, 2022}{Managing scope ambiguities in Glue via multistage
  proving. Joint work with Dag T. T. Haug. Presented at the 32nd South of
  England LFG Meeting, University of Oxford.
  \slides{findlay-haug-selfg32-slides-handout.pdf}}

\Reff*{sin22}{4 May, 2022}{Universal Natural Language Understanding: combining
  Universal Dependencies, Glue Semantics, and machine learning. Joint work with
  Dag T. T. Haug. Presented at the 1st Semantics in Norway (SiN) workshop,
  Helgaker. \slides{findlay-haug-sin22-slides-handout.pdf}}

\Reff*{depling2021}{10 Dec., 2021}{Defrosting frozen metaphors: idiom extensions
  and the proper representation of idiomatic expressions. Presented at the
  Linguistics Departmental Seminar (Spr\aa kvitenskapelig Instituttseminar),
  Department of Linguistics and Scandinavian Studies, University of Oslo.
  \slides{dep-ling-sem-dec-2021.pdf}}

\Reff*{fmm2018}{4 Aug., 2018}{Pulling a pretence rabbit out of the hat. Joint
  work with Sascha Bargmann and Manfred Sailer. Presented (by Manfred Sailer) at
  the Workshop on Form-Meaning Mismatches in Natural Language (FMM2018),
  Georg-August-Univers\"{a}t G\"{o}ttingen. \slides{fmm2018.pdf}}

\Reff*{selfg25}{26 May, 2018}{Lexicalised LFG. Presented at the 25th South of
  England LFG Meeting, SOAS, London. \handout{se-lfg25.pdf}}

\Reff*{selfg23}{13 May, 2017}{When the subject honorific brings honour to all:
  the expanding role of pragmatics in Korean honorification. Joint work with
  Yoolim Kim. Presented at the 23rd South of England LFG Meeting, SOAS, London.
  \slides{kim_findlay-SELFG23-handout_slides.pdf}}

\Reff*{selfg22}{4 Feb., 2017}{Multiword expressions and lexicalism. Presented at
  the 22nd South of England LFG Meeting, SOAS, London.
  \handout{SE-LFG_Feb2017-1up.pdf}}

\Reff*{selfg19}{20 Feb., 2016}{Idioms in LFG. Presented at the 19th South of
  England LFG Meeting, SOAS, London. \handout{SELFG_Presentation_idioms.pdf}}

\Reff*{selfg15}{1 Nov., 2014}{Mapping theory without argument structure.
  Presented at the 15th South of England LFG Meeting, SOAS, London.
  \handout{SELFG_Presentation_mapping-theory.pdf}}

\Reff*{mphil-prs}{22 Oct., 2013}{The pseudopassive: a Lexical Functional
  account. Presented at the MPhil/PRS Thesis Workshop, University of Oxford.
  \handout{MPhil_thesis_workshop_presentation.pdf}}

\Reff*{swg-tt2013}{8 May, 2013}{The pseudopassive in LFG: some preliminary
  thoughts. Presented at the Syntax Working Group, University of Oxford.
  \handout{P-passive_SWG.pdf}}

\end{document}
