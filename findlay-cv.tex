%==============================================================================%
%                                                                              %
%  file:   findlay-cv.tex                                                      %
%  author: Jamie Findlay <jy.findlay@gmail.com>                                %
%  With thanks to Brian Buccola <brian.buccola@gmail.com> for the template     %
%                                                                              %
%==============================================================================%

%============%
%            %
%  PREAMBLE  %
%            %
%============%

\documentclass[11pt,a4paper,twoside]{article}

%===============================%
%  Packages and basic settings  %
%===============================%

\newcommand{\name}{Jamie Y. Findlay}

% Font settings
\usepackage[T1]{fontenc}
\usepackage[utf8]{inputenc}
\usepackage{microtype}
\usepackage[sc,osf]{mathpazo}% Palatino font, w/ smallcaps & old-style figures
\linespread{1.05}
\usepackage{marvosym}% provides \Letter and \Telefon

% Margins
\usepackage[margin=1.5in,twoside,centering,marginparsep=0.8cm]{geometry}

% Section fonts
\usepackage{titlesec}
\titleformat*{\section}{\normalfont\Large\scshape}
\titleformat*{\subsection}{\normalfont\large\bfseries}
\titleformat*{\subsubsection}{\normalfont\normalsize\itshape}

% Headers, footers, and page styles
\usepackage{titleps}
\newpagestyle{main}[\footnotesize]{%
  \headrule
  \sethead[Curriculum Vit\ae][][Jamie Y. Findlay]
          {Curriculum Vit\ae}{}{Jamie Y. Findlay}
  \setfoot[][\thepage][\scriptsize \today]
          {}{\thepage}{\scriptsize \today}
}
\newpagestyle{first}[\footnotesize]{%
  \sethead{}{}{}
  \setfoot{}{}{\scriptsize \today}
}
\pagestyle{main}

% Packages for CV environment
\usepackage{array}% required for defining newcolumntype with custom vrule
\usepackage{longtable}% normal \tabular environment does not allow page breaks
\setlength{\LTpre}{0pt}% glue before longtable
\addtolength{\LTpost}{0pt}% glue after longtable

% Hyperlinks
\usepackage[breaklinks]{hyperref}
\usepackage[hyphenbreaks]{breakurl}
\usepackage{graphicx}
\setlength{\parindent}{0pt}

\usepackage{marginnote}

%Set some colours
\usepackage[svgnames]{xcolor}
\definecolor{gold}{HTML}{c69c6e}
\definecolor{green}{HTML}{008000}
% \definecolor{green}{HTML}{39B54A}

% Force marginnotes to be on the LHS
\usepackage{etoolbox}
\makeatletter
\patchcmd{\@mn@margintest}{\@tempswafalse}{\@tempswatrue}{}{}
\patchcmd{\@mn@margintest}{\@tempswafalse}{\@tempswatrue}{}{}
\reversemarginpar
\makeatother

%==========%
%  Macros  %
%==========%

\newcommand{\datewidth}{0.20}
\newcommand{\bodywidth}{0.77}

\newcommand{\oa}[2]{\marginnote{[\textcolor{#1}{\href{#2}{\texttt{paper}}}]}} % For open access links

\newcommand{\marginlink}[2]{\marginnote{\href{http://users.ox.ac.uk/~sjoh2787/#2}{\texttt{[#1]}}}}

\newcommand{\handout}[1]{\marginlink{handout}{#1}}
\newcommand{\poster}[1]{\marginlink{poster}{#1}}
\newcommand{\slides}[1]{\marginlink{slides}{#1}}
% \newcommand{\handout}[1]{\marginnote{\href{http://users.ox.ac.uk/~sjoh2787/#1}{\texttt{[handout]}}}}
% \newcommand{\poster}[1]{\marginnote{\href{http://users.ox.ac.uk/~sjoh2787/#1}{\texttt{[poster]}}}}
% \newcommand{\slides}[1]{\marginnote{\href{http://users.ox.ac.uk/~sjoh2787/#1}{\texttt{[slides]}}}}


\newcolumntype{L}{>{\raggedright}p{\datewidth\textwidth}}
\newcolumntype{R}{p{\bodywidth\textwidth}}

\newenvironment{cvsection}{%
  \setlength{\extrarowheight}{0.70ex}
  \begin{longtable}[l]{@{} L R @{}}
  % Comment line above and uncomment line below to add gray vrule between date and body.
  % \begin{longtable}{@{} L !{\myvrule} R @{}}
}{%
  \end{longtable}
}


\frenchspacing

%========%
%        %
%  BODY  %
%        %
%========%

\begin{document}

\thispagestyle{first}

\begin{center}
  {\LARGE Curriculum Vit\ae\\
  \Huge\textbf{Jamie Y. Findlay}\\}
\end{center}

\vspace{1em}

\section*{Contact Information}
\begin{minipage}[t]{0.5\textwidth}
 Centre for Linguistics and Philology,  \\
Walton St., Oxford, OX1 2HG\\
United Kingdom
\end{minipage}
\begin{minipage}[t]{0.55\textwidth}
 % \Telefon\ +1 312-945-6857 {\footnotesize (US)}\\
  \Letter\  \href{mailto:jamie.findlay@ling-phil.ox.ac.uk}{\nolinkurl{jamie.findlay@ling-phil.ox.ac.uk}}\\
  \Keyboard\ \url{http://jyfindlay.com}
\end{minipage}

\section*{Research Interests}

Syntax, semantics, grammatical theory and the syntax-semantics interface (especially Lexical Functional Grammar and Glue Semantics), philosophy of language, (critical) discourse analysis, language \& gender, language \& sexuality.

\section*{Education}

\begin{cvsection}
  2015-- & \parbox[t]{\bodywidth\textwidth}{%
    D.Phil. in Comparative Philology \& General Linguistics, \\University of Oxford.\\
    {\footnotesize Thesis title: \emph{Multiword expressions and the lexicon: a TAG-LFG approach}.}\\
    {\footnotesize Supervisors: Ash Asudeh, Mary Dalrymple.}
  }\\
  2012--2014 & \parbox[t]{\bodywidth\textwidth}{%
    M.Phil. (with distinction) in General Linguistics \& Comparative Philology, University of Oxford.\\
    {\footnotesize Thesis title: \emph{The prepositional passive: a Lexical Functional account}.}\\
    {\footnotesize Supervisor: Mary Dalrymple.}
  }\\
  2011 & \parbox[t]{\bodywidth\textwidth}{%
    Graduate Programme in Linguistics, McGill University.
  }\\
  2007--2011 & \parbox[t]{\bodywidth\textwidth}{%
    B.A. (with First Class Honours) in Modern Languages (French) \& Linguistics, University of Oxford.
  }
\end{cvsection}

\section*{Teaching Qualifications}

\begin{cvsection}
  2017\marginlink{portfolio}{portfolio-JYF-submission.pdf} & \parbox[t]{\bodywidth\textwidth}{%
    SEDA-PDF Supporting Learning award. \\
    {\footnotesize (Mapped to Descriptor 1 of the UK Professional Standards Framework: equivalent to Associate Fellow of the Higher Education Academy.)}%
    }
\end{cvsection}

\section*{Employment}

\begin{cvsection}
  Oct. 2014-- & Tutor, Faculty of Linguistics, Philology \& Phonetics, University of Oxford.\\
  July 2012--	& Freelance linguistic consultant on various NLP tasks.\\
  % Nov.--Dec. 2016	& Volunteer annotator, PARSEME shared task on automatic identification of verbal multiword expressions.\\
  \parbox[t]{\datewidth\textwidth}{Dec. 2014--\\\hspace*{1em}Oct. 2015}	& Research Assistant, Oxford English Dictionary, Oxford University Press.\\
  Feb.--Sept. 2014	&  Annotator, Supportive Automated Feedback for Short Essay Answers (SAFeSEA) project, Department of Computer Science, University of Oxford.\\
  May--Sept. 2013	& Casual Research Assistant to Mary Dalrymple, Faculty of Linguistics, Philology \& Phonetics, University of Oxford.\\
\end{cvsection}

\section*{Scholarship}
\textit{Note:} To the left of each item below, I provide a link to an accessible version. When that is a published version, I give an indication of its Open Access status by making the text \textcolor{gold}{gold} or \textcolor{green}{green}.

\subsection*{Articles in Refereed Journals and Books}

\begin{cvsection}
  In preparation &  Mary Dalrymple and Jamie Y. Findlay. Lexical Functional Grammar. In Andr\'{a}s Kert\'{e}sz, Edith Moravcsik, and Csilla R\'{a}kosi (eds.), \emph{Current approaches to syntax -- a comparative handbook}. De Gruyter Mouton.\\

  2017\oa{green}{http://users.ox.ac.uk/~sjoh2787/house_of_lords.pdf} 	& Jamie Y. Findlay. Unnatural acts lead to unconsummated marriages: discourses of homosexuality within the {House of Lords} debate on same-sex marriage. \emph{Journal of Language and Sexuality} 6(1), 30--60.\\

  2016\oa{gold}{http://jlm.ipipan.waw.pl/index.php/JLM/article/view/171/148} & Jamie Y. Findlay. Mapping theory without argument structure. \emph{Journal of Language Modelling} 4(2), 293--338.
  % \url{http://dx.doi.org/10.15398/jlm.v4i2.171}.
   \\

  2013\oa{green}{http://users.ox.ac.uk/~sjoh2787/medical_terminology.pdf} & Janak Bechar, Jamie Y. Findlay, and Joseph Hardwicke. Understanding medical words of Greek and Latin origin. \emph{Student BMJ} 21.
\end{cvsection}

\subsection*{Articles in Refereed Conference Proceedings}

\begin{cvsection}
  Forthcoming & Jamie Y. Findlay. Multiword expressions and lexicalism. In Miriam Butt and Tracy Holloway King (eds.), \emph{Proceedings of the LFG17 Conference}. Stanford, CA: CSLI Publications.\\

  2017\oa{gold}{http://aclweb.org/anthology/W17-1709} & Jamie Y. Findlay. Multiword expressions and lexicalism: the view from LFG. In \emph{Proceedings of the 13th Workshop on Multiword Expressions (MWE 2017)}, 73--79. Association for Computational Linguistics.\\

  2016\oa{gold}{http://web.stanford.edu/group/cslipublications/cslipublications/HPSG/2016/headlex2016-findlay.pdf} & Jamie Y. Findlay. The prepositional passive in Lexical Functional Grammar. In Doug Arnold, Miriam Butt, Berthold Crysmann, Tracy Holloway King, and Stefan M\"uller (eds.), \emph{Proceedings of the Joint 2016 Conference on Head-driven Phrase Structure Grammar and Lexical Functional Grammar}, 255--275. Stanford, CA: CSLI Publications.
\end{cvsection}

\subsection*{Refereed Presentations}

\begin{cvsection}
  6 Sept., 2017\slides{kim-findlay-lagb.pdf} & The logic of Korean honorification. Joint work with Yoolim Kim. Presented at the 2017 Annual Meeting of the Linguistics Association of Great Britain (LAGB 2017), University of Kent, Canterbury.\\

  27 July, 2017\handout{lfg17_handout.pdf} & Multiword expressions and lexicalism. Presented at the 22nd International Lexical Functional Grammar Conference, University of Konstanz.\\

  15 June, 2017\slides{kim_findlay-LangUE2017-handout.pdf} & When the subject honorific brings honour to all: the expanding role of pragmatics in Korean honorification. Joint work with Yoolim Kim. Presented at the 12th Language at the University of Essex Postgraduate Conference (LangUE 2017), Colchester.\\

  4 Apr., 2017\poster{eacl_poster.pdf} & Multiword expressions and lexicalism: the view from LFG. Poster presented at the 13th Workshop on Multiword Expressions (MWE 2017), Valencia.\\

  29 July, 2016\handout{HeadLex_Presentation.pdf} & The prepositional passive in Lexical Functional Grammar. Presented at HeadLex16, Polish Academy of Sciences, Warsaw.\\

  5 June, 2014\handout{IGALA8Presentation.pdf} & Unnatural acts lead to unconsummated marriages: discourses of homosexuality in the House of Lords revisited. Presented at the 8th International Gender and Language Association Conference, Simon Fraser University, Vancouver.\\
 \end{cvsection}

\subsection*{Invited Talks}

\begin{cvsection}
  31 Jan., 2017\handout{SOAS_Presentation_p-passive.pdf} & The pseudopassive: not so phoney after all. Presented at the Linguistics Departmental Seminar Series, School of Oriental and African Studies (SOAS), London.\\
\end{cvsection}
\newpage
\subsection*{Other Presentations}

\begin{cvsection}

  13 May, 2017\slides{kim_findlay-SELFG23-handout_slides.pdf} & When the subject honorific brings honour to all: the expanding role of pragmatics in Korean honorification. Joint work with Yoolim Kim. Presented at the 23rd South of England LFG Meeting, SOAS, London.\\

  4 Feb., 2017\handout{SE-LFG_Feb2017-1up.pdf} & Multiword expressions and lexicalism. Presented at the 22nd South of England LFG Meeting, SOAS, London.\\

  20 Feb., 2016\handout{SELFG_Presentation_idioms.pdf}	& Idioms in LFG. Presented at the 19th South of England LFG Meeting, SOAS, London.\\

  1 Nov., 2014\handout{SELFG_Presentation_mapping-theory.pdf}	& Mapping theory without argument structure. Presented at the 15th South of England LFG Meeting, SOAS, London.\\

  22 Oct., 2013\handout{MPhil_thesis_workshop_presentation.pdf}	& The pseudopassive: a Lexical Functional account. Presented at the MPhil/PRS Thesis Workshop, University of Oxford.\\

  8 May, 2013\handout{P-passive_SWG.pdf}	& The pseudopassive in LFG: some preliminary thoughts. Presented at the Syntax Working Group, University of Oxford.
\end{cvsection}

\section*{Teaching}

\subsection*{University of Oxford}
\textit{Note:} At Oxford, `Prelims' = first year undergraduate level; `FHS (Final Honour School)' = advanced undergraduate level.

\subsubsection*{Lectures}
\begin{cvsection}
2018    & Prelims Semantics \& Pragmatics.\\
2016--  & Occasional lectures in syntax (undergraduate and graduate level) and `formal foundations of linguistics' (graduate level), filling in for lecturer absences.
\end{cvsection}

\subsubsection*{Graduate classes}
\begin{cvsection}
2017        & Formal foundations of linguistics -- group size: 23.\\
2015--2016	&	Introduction to syntax -- group size: 25.\\
2014			  & Introduction to phonology -- group size: 12.
\end{cvsection}

\subsubsection*{Undergraduate classes}
\begin{cvsection}
2014			  &	Prelims Grammatical Analysis -- group size: 6.
\end{cvsection}
\newpage
\subsubsection*{Tutorials}

\begin{cvsection}
2017--			&	FHS Sociolinguistics.\\
2014--			&	FHS Syntax.\\
2014--			&	FHS Semantics \& Pragmatics.\\
2014				&	Prelims General Linguistics, Psycholinguistics, Historical Linguistics, Semantics \& Pragmatics, and Sociolinguistics.\\
\end{cvsection}

\subsubsection*{Miscellaneous}

\begin{cvsection}
%\parbox[t]{\datewidth\textwidth}{15 \& 22 Feb.,\\\hspace*{0.5em} 2017}		& Two \LaTeX{} workshops for linguistics graduate students.\\
	15 Feb., 2017\handout{latex-basics.pdf}		& \parbox[t]{\bodywidth\textwidth}{Workshop: `\LaTeX{} basics'.}\\
  %
	22 Feb., 2017\handout{latex-advanced-topics.pdf}		& \parbox[t]{\bodywidth\textwidth}{Workshop: `Advanced topics in \LaTeX'.\\
  {\footnotesize(Two workshops to introduce graduate students in linguistics to \LaTeX.)}}\\
  %
	25 Jan., 2017\handout{Socio_presentation.pdf}		& \parbox[t]{\bodywidth\textwidth}{`Language and identity: the view from linguistics'.\\{\footnotesize(An introduction to sociolinguistics for first year English language and literature undergraduates.)}}

\end{cvsection}

\section*{Honours and Awards}

\begin{cvsection}
  2015--2017 	& Professor Paul Slack Scholarship,
  Linacre College, University of Oxford.\\
  %
  2015--2017 	& AHRC Studentship, Faculty of
  Linguistics, Philology \& Phonetics, University of Oxford.\\
  %
  July 2014	& George Wolf Prize for best overall
  performance in the M.Phil. or M.St. degree, Faculty of Linguistics,
  Philology \& Phonetics, University of Oxford.\\
  %
  June 2014	& Graduate student essay prize,
  International Gender and Language Association.\\
  %
  2012--2014	& AHRC Studentship, Faculty of
  Linguistics, Philology \& Phonetics, University of Oxford.\\
  %
  2011		& McCall McBain Fellowship, McGill University.\\
  %
  2008--2011	& Casberd Scholarship, St.\ John's
  College, University of Oxford.\\
\end{cvsection}

\section*{Service}

\subsection*{To the department}

\begin{cvsection}
  2017--      & Co-founder and co-convenor, DPhil workshop, Faculty of Linguistics, Philology \& Phonetics, University of Oxford.\\
  2015--      & Admissions interviewing for undergraduate degrees involving linguistics.\\
  2017 		    & Convenor (Hilary Term), Syntax Working Group, University of Oxford.\\
  2009--2010	& President, Oxford University Linguistics Society.\\
  2007--2009	& Student representative, Faculty of Linguistics, Philology \& Phonetics, University of Oxford.\\
\end{cvsection}

\subsection*{To the profession}

% \subsubsection*{Journal reviewing}
% Journal of Linguistics
%
% \subsubsection*{Conference reviewing}
% Conference of the Student Organisation of Linguistics in Europe (ConSOLE)
%
% \subsubsection*{Other}
% Volunteer copy editor for \textit{Journal of Language Modelling}.

\begin{cvsection}
Reviewer (journals)	&	\textit{Journal of Linguistics} (2015, 2017).\\
Reviewer (conferences) & \textit{Conference of the Student Organisation of Linguistics in Europe (ConSOLE)} (2017).\\
Marker    & UK Linguistics Olympiad (2014--).\\
Copy editor& \textit{Journal of Language Modelling}.
\end{cvsection}

\section*{Memberships}

\begin{cvsection}
  2017--	& Association for Computational Linguistics (ACL).\\
  2014--  & International Lexical Functional Grammar Assoc. (ILFGA).\\
  2012-- 	& Canadian Linguistic Association (CLA\slash ACL).\\
  2011-- 	& Linguistics Association of Great Britain (LAGB).\\
  2011-- 	& International Gender and Language Association (IGALA).\\
\end{cvsection}

\section*{Languages}

\begin{cvsection}
  \textbf{Natural}      & English~(native), French~(near-fluent), Latin~(intermediate), German~(basic). \\
  \textbf{Programming}  & Bash (Unix shell), Python. \\
  \textbf{Markup}       & \LaTeX\slash Bib\TeX, HTML\slash CSS.
\end{cvsection}

\end{document}
