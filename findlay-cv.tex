%==============================================================================%
%                                                                              %
%  file:   findlay-cv.tex                                                      %
%  author: Jamie Findlay <jy.findlay@gmail.com>
%  With thanks to Brian Buccola <brian.buccola@gmail.com> for the template                           %
%                                                                              %
%==============================================================================%

%============%
%            %
%  PREAMBLE  %
%            %
%============%

\documentclass[11pt,a4paper,twoside]{article}

%===============================%
%  Packages and basic settings  %
%===============================%

\newcommand{\name}{Jamie Y. Findlay}

% Font settings
\usepackage[T1]{fontenc}
\usepackage[utf8]{inputenc}
\usepackage{microtype}
\usepackage[sc,osf]{mathpazo}% Palatino font, w/ smallcaps & old-style figures
\linespread{1.05}
\usepackage{marvosym}% provides \Letter and \Telefon

% Margins
\usepackage[margin=1.5in,twoside,centering,marginparsep=0.8cm]{geometry}

% Section fonts
\usepackage{titlesec}
\titleformat*{\section}{\normalfont\Large\scshape}
\titleformat*{\subsection}{\normalfont\large\bfseries}
\titleformat*{\subsubsection}{\normalfont\normalsize\itshape}

% Headers, footers, and page styles
\usepackage{titleps}
\newpagestyle{main}[\footnotesize]{%
  \headrule
  \sethead[Curriculum Vit\ae][][Jamie Y. Findlay]
          {Curriculum Vit\ae}{}{Jamie Y. Findlay}
  \setfoot[][\thepage][\scriptsize \today]
          {}{\thepage}{\scriptsize \today}
}
\newpagestyle{first}[\footnotesize]{%
  \sethead{}{}{}
  \setfoot{}{}{\scriptsize \today}
}
\pagestyle{main}

% References
%\usepackage[backend=bibtex,style=numeric,maxnames=99]{biblatex}
%\addbibresource{/home/brian/repos/references/references.bib}
%\renewcommand{\subtitlepunct}{\addcolon\space}
%% Remove default quotes around title.
%\DeclareFieldFormat[article,inproceedings,unpublished]{citetitle}{#1}
%\DeclareFieldFormat[article,inproceedings,unpublished]{title}{#1}
%\DeclareFieldFormat[thesis]{citetitle}{\emph{#1}}
%\DeclareFieldFormat[thesis]{title}{\emph{#1}}
%% Don't print pubstate field.
%\AtEveryCitekey{\clearfield{pubstate}}

% Packages for CV environment
\usepackage{array}% required for defining newcolumntype with custom vrule
\usepackage{longtable}% normal \tabular environment does not allow page breaks
\setlength{\LTpre}{0pt}% glue before longtable
\addtolength{\LTpost}{0pt}% glue after longtable

% Hyper links and PDF info
\usepackage[breaklinks]{hyperref}
%\hypersetup{%
%  colorlinks  = false,
%  pdfauthor   = {\name},
%  pdftitle    = {Curriculum Vitae -- \name},
%  pdfsubject  = {Curriculum Vitae},
%  pdfkeywords = {\name, CV, linguistics},
%  pdfpagemode = UseNone
%}
\usepackage[hyphenbreaks]{breakurl}
\usepackage{graphicx}
\setlength{\parindent}{0pt}

\usepackage{marginnote}

%Set some colours
\usepackage[svgnames]{xcolor}
\definecolor{gold}{HTML}{c69c6e}
\definecolor{green}{HTML}{008000}
% \definecolor{green}{HTML}{39B54A}

% Force marginnotes to be on the RHS
\usepackage{etoolbox}

\makeatletter
\patchcmd{\@mn@margintest}{\@tempswafalse}{\@tempswatrue}{}{}
\patchcmd{\@mn@margintest}{\@tempswafalse}{\@tempswatrue}{}{}
\reversemarginpar
\makeatother

%==========%
%  Macros  %
%==========%

\newcommand{\datewidth}{0.20}
\newcommand{\bodywidth}{0.77}

\newcommand{\oa}[2]{\marginnote{[\textcolor{#1}{\href{#2}{\texttt{paper}}}]}} % For open access links
\newcommand{\handout}[1]{\marginnote{\href{http://users.ox.ac.uk/~sjoh2787/#1}{\texttt{[handout]}}}}
\newcommand{\poster}[1]{\marginnote{\href{http://users.ox.ac.uk/~sjoh2787/#1}{\texttt{[poster]}}}}

\newcolumntype{L}{>{\raggedright}p{\datewidth\textwidth}}
\newcolumntype{R}{p{\bodywidth\textwidth}}

\newenvironment{cvsection}{%
  \setlength{\extrarowheight}{0.70ex}
  \begin{longtable}[l]{@{} L R @{}}
  % Comment line above and uncomment line below to add gray vrule between date and body.
  % \begin{longtable}{@{} L !{\myvrule} R @{}}
}{%
  \end{longtable}
}


\frenchspacing

%========%
%        %
%  BODY  %
%        %
%========%

\begin{document}

\thispagestyle{first}

\begin{center}
  {\LARGE Curriculum Vit\ae\\
  \Huge\textbf{Jamie Y. Findlay}\\}
\end{center}

\vspace{1em}

%\textbf{Citizenship:} British \hfill \textbf{Languages:} English (native), French (advanced), German (beginner)

\section*{Contact Information}
\begin{minipage}[t]{0.5\textwidth}
 Centre for Linguistics and Philology,  \\
Walton St., Oxford, OX1 2HB\\
United Kingdom
\end{minipage}
\begin{minipage}[t]{0.55\textwidth}
 % \Telefon\ +1 312-945-6857 {\footnotesize (US)}\\
  \Letter\  \href{mailto:jamie.findlay@ling-phil.ox.ac.uk}{\nolinkurl{jamie.findlay@ling-phil.ox.ac.uk}}\\
  \Keyboard\ \url{http://jyfindlay.com}
\end{minipage}

\section*{Research Interests}

Syntax, semantics, grammatical theory and the syntax-semantics interface (especially Lexical Functional Grammar and Glue Semantics), philosophy of language, (critical) discourse analysis, language \& gender, language \& sexuality.

\section*{Education}

\begin{cvsection}
  2015-- & \parbox[t]{\bodywidth\textwidth}{%
    D.Phil. in Comparative Philology \& General Linguistics, \\University of Oxford.\\
    {\footnotesize Thesis title: \emph{Multiword expressions and the lexicon: a TAG-LFG approach}}\\
    {\footnotesize Supervisors: Mary Dalrymple, Ash Asudeh}
  }\\
  2012--2014 & \parbox[t]{\bodywidth\textwidth}{%
    M.Phil. (with distinction) in General Linguistics \& Comparative Philology, University of Oxford.
  }\\
  2011 & \parbox[t]{\bodywidth\textwidth}{%
    Graduate Programme in Linguistics, McGill University.
  }\\
  2007--2011 & \parbox[t]{\bodywidth\textwidth}{%
    B.A. (with First Class Honours) in Modern Languages (French) \& Linguistics, University of Oxford.
  }
\end{cvsection}

\section*{Employment}

\begin{cvsection}
  Oct. 2014-- & Tutor, Faculty of Linguistics, Philology \& Phonetics, University of Oxford.\\
  July 2012--	& Freelance linguistic consultant on various NLP tasks.\\
  Nov.--Dec. 2016	& Volunteer annotator, PARSEME shared task on automatic identification of verbal multiword expressions.\\
  \parbox[t]{\datewidth\textwidth}{Dec. 2014--\\\hspace*{1em}Oct. 2015}	& Research Assistant, Oxford English Dictionary, Oxford University Press.\\
  Feb.--Sept. 2014	&  Annotator, Supportive Automated Feedback for Short Essay Answers (SAFeSEA) project, Department of Computer Science, University of Oxford.\\
  May--Sept. 2013	& Casual Research Assistant to Mary Dalrymple, Faculty of Linguistics, Philology \& Phonetics, University of Oxford.\\
\end{cvsection}

\section*{Scholarship}
\textit{Note:} To the left of each item below, I provide a link to an accessible version. When that is a published version, I give an indication of its Open Access status by making the text \textcolor{gold}{gold} or \textcolor{green}{green}.

\subsection*{Articles in Refereed Journals}

\begin{cvsection}
  In prep.\oa{green}{https://oxris.ox.ac.uk/repository.html?pub=687885\&file-url=http\%3A\%2F\%2Fsymplectic.ora.ox.ac.uk\%3A8080\%2Fpubs\%2Ffile\%2Finfo\%3Afedora\%2Fuuid\%3A70526a31-e58b-4680-81f3-7ee2187856d6\%2Fbinb2bf7db1-d518-4106-9b92-e482a6f3619c\%2Fhouse_of_lords.pdf} 	& Jamie Y. Findlay. Unnatural acts lead to unconsummated marriages: discourses of homosexuality within the {House of Lords} debate on same-sex marriage. (Scheduled to appear in 2017 in \emph{Journal of Language and Sexuality} 6(2).)\\

  2013\oa{green}{https://oxris.ox.ac.uk/repository/files/?rep=1\&pub=689106\&file-url=http\%3A\%2F\%2Fsymplectic.ora.ox.ac.uk\%3A8080\%2Fpubs\%2Ffile\%2Finfo\%3Afedora\%2Fuuid\%3A274b8811-cf64-40e5-bb44-9e24c4de411d\%2Fbin6e2848f3-3b04-47f5-89a7-b51c8621923a\%2Fmedical_terminology.pdf} & Janak Bechar, Jamie Y. Findlay, and Joseph Hardwicke. Understanding medical words of Greek and Latin origin. \emph{Student BMJ} 21.
\end{cvsection}

\subsection*{Articles in Refereed Conference Proceedings}

\begin{cvsection}
  2017\oa{gold}{http://aclweb.org/anthology/W17-1709} & Jamie Y. Findlay. Multiword expressions and lexicalism: the view from LFG. In \emph{Proceedings of the 13th Workshop on Multiword Expressions (MWE 2017)}, 73--79. Association for Computational Linguistics.\\

  2016\oa{gold}{http://web.stanford.edu/group/cslipublications/cslipublications/HPSG/2016/headlex2016-findlay.pdf} & Jamie Y. Findlay. The prepositional passive in Lexical Functional Grammar. In Doug Arnold, Miriam Butt, Berthold Crysmann, Tracy Holloway King, and Stefan M\"uller (eds.), \emph{Proceedings of the Joint 2016 Conference on Head-driven Phrase Structure Grammar and Lexical Functional Grammar}, 255--275. Stanford, CA: CSLI Publications.
\end{cvsection}

\subsection*{Refereed Presentations}

\begin{cvsection}
  4 April, 2017\poster{eacl_poster.pdf} & Multiword expressions and lexicalism: the view from LFG. Poster presented at the 13th Workshop on Multiword Expressions (MWE 2017), Valencia.\\

  29 July, 2016\handout{HeadLex_Presentation.pdf} & The prepositional passive in Lexical Functional Grammar. Presented at HeadLex16, Polish Academy of Sciences, Warsaw.\\

  5 June, 2014\handout{IGALA8Presentation.pdf} & Unnatural acts lead to unconsummated marriages: discourses of homosexuality in the House of Lords revisited. Presented at the 8th International Gender and Language Association Conference, Simon Fraser University, Vancouver.\\
 \end{cvsection}

\subsection*{Invited Talks}

\begin{cvsection}
  31 Jan., 2017\handout{SOAS_Presentation_p-passive.pdf} & The pseudopassive: not so phoney after all. Presented at the Linguistics Departmental Seminar Series, School of Oriental and African Studies (SOAS), London.\\
\end{cvsection}

\subsection*{Other Presentations}

\begin{cvsection}
  4 Feb., 2017\handout{SE-LFG_Feb2017-1up.pdf} & Multiword expressions and lexicalism. Presented at the 22nd South of England LFG Meeting, SOAS, London.\\

  20 Feb., 2016\handout{SELFG_Presentation_idioms.pdf}	& Idioms in LFG. Presented at the 19th South of England LFG Meeting, SOAS, London.\\

  1 Nov., 2014\handout{SELFG_Presentation_mapping-theory.pdf}	& Mapping theory without argument structure. Presented at the 15th South of England LFG Meeting, SOAS, London.\\

  22 Oct., 2013\handout{MPhil_thesis_workshop_presentation.pdf}	& The pseudopassive: a Lexical Functional account. Presented at the MPhil/PRS Thesis Workshop, University of Oxford.\\

  8 May, 2013\handout{P-passive_SWG.pdf}	& The pseudopassive in LFG: some preliminary thoughts. Presented at the Syntax Working Group, University of Oxford.
\end{cvsection}

\section*{Teaching}

\subsection*{University of Oxford}
\textit{Note:} At Oxford, `Prelims' = first year undergraduate level; `FHS (Final Honour School)' = advanced undergraduate level.

\subsubsection*{Lectures}
\begin{cvsection}
2016--  & Occasional lectures in syntax and `formal foundations', filling in for lecturer absences.
\end{cvsection}

\subsubsection*{Classes}

\begin{cvsection}
2015--2016	&	Introduction to syntax practicals (graduate).\\
2014			&	Introduction to phonology practicals (graduate).\\
2014			&	Prelims Grammatical Analysis (Paper X).
\end{cvsection}
\newpage
\subsubsection*{Tutorials}

\begin{cvsection}
2017--			&	FHS Sociolinguistics (Paper XIII).\\
2014--			&	FHS Syntax (Papers XII \& XIII).\\
2014--			&	FHS Semantics \& Pragmatics (Papers XII \& XIII).\\
2014				&	Prelims General Linguistics, Psycholinguistics, Historical Linguistics, Semantics \& 					Pragmatics, and Sociolinguistics (Paper VIII).\\
\end{cvsection}

\subsubsection*{Miscellaneous}

\begin{cvsection}
%\parbox[t]{\datewidth\textwidth}{15 \& 22 Feb.,\\\hspace*{0.5em} 2017}		& Two \LaTeX{} workshops for linguistics graduate students.\\
	15 Feb., 2017\handout{latex-basics.pdf}		& \parbox[t]{\bodywidth\textwidth}{Workshop: `\LaTeX{} basics'}\\
  %
	22 Feb., 2017\handout{latex-advanced-topics.pdf}		& \parbox[t]{\bodywidth\textwidth}{Workshop: `Advanced topics in \LaTeX'\\
  {\footnotesize(Two workshops to introduce graduate students in linguistics to \LaTeX.)}}\\
  %
	25 Jan., 2017\handout{Socio_presentation.pdf}		& \parbox[t]{\bodywidth\textwidth}{`Language and identity: the view from linguistics'.\\{\footnotesize(An introduction to 							sociolinguistics for first year English language and literature undergraduates.)}}

\end{cvsection}

\section*{Honours and Awards}

\begin{cvsection}
  2015--2017 	& Professor Paul Slack Scholarship,
  Linacre College, University of Oxford.\\
  %
  2015--2017 	& AHRC Studentship, Faculty of
  Linguistics, Philology \& Phonetics, University of Oxford.\\
  %
  July 2014	& George Wolf Prize for best overall
  performance in the M.Phil. or M.St. degree, Faculty of Linguistics,
  Philology \& Phonetics, University of Oxford.\\
  %
  June 2014	& Graduate student essay prize,
  International Gender and Language Association.\\
  %
  2012--2014	& AHRC Studentship, Faculty of
  Linguistics, Philology \& Phonetics, University of Oxford.\\
  %
  2011		& McCall McBain Fellowship, McGill University.\\
  %
  2008--2011	& Casberd Scholarship, St.\ John's
  College, University of Oxford.\\
\end{cvsection}

\section*{Service}

\subsection*{To the department}

\begin{cvsection}
  2017 		& Convenor (Hilary Term), Syntax Working Group, University of Oxford.\\
  2009--2010	& President, Oxford University Linguistics Society.\\
  2007--2009	& Student representative, Faculty of Linguistics, Philology \& Phonetics, University of Oxford.\\
\end{cvsection}

\subsection*{To the profession}

\begin{cvsection}
Reviewer	&	\textit{Journal of Linguistics} (1 journal article manuscript, 2015).
\end{cvsection}

\section*{Associations}

\begin{cvsection}
  2017 {\footnotesize (joined)}	& Association for Computational Linguistics (ACL).\\
  2014 {\footnotesize (joined)}	& International Lexical Functional Grammar Assoc. (ILFGA).\\
  2012 {\footnotesize (joined)} 	& Canadian Linguistic Association (CLA\slash ACL).\\
  2011 {\footnotesize (joined)} 	& Linguistics Association of Great Britain (LAGB).\\
  2011 {\footnotesize (joined)} 	& International Gender and Language Association (IGALA).\\
\end{cvsection}

\section*{Languages}

\begin{cvsection}
  \textbf{Natural}      & English (native), French (advanced), German (beginner). \\
  \textbf{Programming}  & Bash (Unix shell), Python. \\
  \textbf{Markup}       & \LaTeX\slash Bib\TeX, HTML\slash CSS.
\end{cvsection}

\end{document}
